\documentclass{trlnotes}
\usepackage{trmath}
\addcompatiblelayout{commonplace}
\setlayout{commonplace}

\usepackage{trthm}
\usepackage{trphys}
\input{mdefs}
\usepackage[normalem]{ulem}
\usepackage{trsym}
\usepackage{docmute}
\usepackage{luacode}
\usepackage{tikz} 
\usepackage[
backend=biber,
sorting=none,
style=authoryear,
language=russian
]{biblatex}
\addbibresource{bibliography.bib}
\graphicspath{{img/}}
\title{методы вычислений}
\date{10.01.2019}
%\\
%   \small версия от \luaexec{tex.print(os.date("\%d.\%m.\%Y \%X"))},
% до экзамена
% \directlua{tex.print(math.floor(os.difftime(os.time{
% day=10, month=1, year=2019, hour=11 },os.time())/3600))}
% часов}
\author{Лектор: Б.~А.~Самокиш}

\usepackage{luatex85}
\usepackage[all]{xy}
\begin{document}
 
\maketitle
\tableofcontents
\clearpage

\makeatletter
\let \oldxypicxto \xto
\let \xto\relax
\newcommand\xto[1]{\ensuremath \xrightarrow[#1]{}}
\makeatother
\chapter{Однородные дифференциальные уравнения}
\label{chap:ode}
\input{tex/ode/ode}

\chapter{Методы линейной алгебры}
\label{chap:linal}
\documentclass{trlnotes}
\usepackage{trmath}
\addcompatiblelayout{commonplace}
\setlayout{commonplace}
\usepackage{trthm}
\usepackage{trsym} 
\usepackage{trphys}
\input{mdefs}
\usepackage{silence}
\usepackage{tikz}
\WarningFilter{latex}{Reference}
\graphicspath{{../../img/}}
\begin{document}

\paragraph{Устойчивость собственных чисел при возмущении матрицы}
\label{par:lin::eigenstab}

Пусть $A$~"--- линейный оператор $\R^s \to \R^s$, $x, b$~"--- векторы-столбцы в $\R^s$.
Здесь будет столько матриц и векторов, что рисовать шляпы не будем, и так понятно 
кто есть кто.

Какие задачи вообще можно здесь решать
\begin{enumerate}
  \item Решение линейной системы $Ax = b$
  \item Поиск собственных чисел $Ax = λx$
\end{enumerate}

Какие при этом могут возникнуть ошибки
\begin{enumerate}
  \item Ошибки округления (алгоритма)
  \item Ошибки начальных данных (неустранимые)
\end{enumerate}

Посмотрим, как оценить ошибки вычисления. 
Пусть $\circ$~--- какая-то операция, а $\circledcirc$~"---  её машинное представление.
Существуют два подхода
\begin{enumerate}
  \item \emph{Прямой анализ ошибок}\par
    Просто учитываем погрешность $a \circ b$ как ошибку округления. Часто
    делают так ($ε_M$~"--- <<машинный эпсилон>>):
    \[
      a \circledcirc b = a \circ b \, (1 + ε), \quad ε \leqslant ε_M
    \]
  \item \emph{Обратный анализ ошибок}(метод эквивалентных возмущений)\par
    Сводим все ошибки к возмущениям начальных данных:
    \[
      a \circledcirc b = \ov~a \circ \ov~b, \quad \ov~a = a + Δa, \; \ov~b = b + Δb.
    \]
    \begin{enumerate}
      \item оцениваем эквивалентные возмущения
      \item оцениваем влияние возмущений
    \end{enumerate}
    получается, что мы все ошибки записали в неустранимые погрешности начальных
    данных
\end{enumerate}

Первый метод частно выдает неправданно большие оценки погрешности,
так что займёмся в основном вторым.

Разберёмся с корректностью задач.
\clause{Решение ЛСУ} ${}$

\begin{defn}[мера обусловленности]\label{defn:lin::eigenstab::condnum}
  $μ = \norm{A}\, \norm{A^{\smash{-1}}}$
\end{defn}
Почему она так выглядит?
Посмотрим какие вообще есть способы оценки вырожденности $A$
\begin{enumerate}
  \item $\det A$. Почти не бывает равным 0. К тому же, перемешивает большие и маленькие 
    собственные числа.
  \item $\dfrac{\norm{Ax}}{\norm{x}}$. Здесь мы пытаемся смотреть на ЛЗ строчек матрицы.
    Но не очень понятно с чем сравнить, чтобы понять близость к ЛЗ. Может быть компоненты
    матрицы маленькие.
  \item $\dfrac{\max \frac{\norm {Ax}}{\norm {x}}}{\min \frac{\norm{Ax}}{\norm{x}}}$ уже выглядит
    разумно. Преобразуем, используя определение нормы (конечномерного) оператора
    \[
      \begin{aligned}
        \max \frac{\norm {Ax}}{\norm {x}} &= \norm{A} \\
        \min \frac{\norm {Ax}}{\norm {x}} &= \min \frac{\norm {y}}{\norm {A^{\smash{-1}}y}} = 
        \frac{1}{\frac{\norm{A^{\smash{-1}}y}}{\norm{y}}} = \norm{A^{{-1}}}^{-1}.
      \end{aligned}
    \]
    А это очень похоже на определение выше.
\end{enumerate}

\begin{lem}\label{lem:lin::eigenstab::idaddinv}
  $\norm{B} < 1\so \exists\,  (I-B)^{-1} \land \norm{(I-B)^{-1}}  \leqslant \dfrac{1}{1 -\norm{B}}$ 
\end{lem}
\begin{prf}
  Рассмотрим систему $x - Bx = y$. Будем искать решение методом простой итерации:
  $x_{n+1} = f(x_n) = B x_n + y$. Покажем, что он сходится. Для этого нужно убедиться что 
  $f$~"---  сжимающее отображение.
  \[
    \norm{f(x) - f(x')} = \norm{B(x-x')} \leqslant \norm{B} \norm{x-x'} < \norm{x-x'}
  \]
  Решение нашлось $\forall\, y \so \exists\, (I-B)^{-1}$.
  Теперь получим оценку нормы
  \[
    \forall\,x \holds x = Bx + y \so \norm{x} \leqslant \norm{B}\norm{x} + \norm{y} 
    \so \norm{x} \leqslant \dfrac{1}{1 -\norm{B}} \norm{y}
  \]
  Тогда это верно и для $\max \norm{x}/\norm{y} = \norm{(I-B)^{-1}}$
\end{prf}

Теперь, оценим, наконец, погрешность решения СЛУ.
\begin{thrm}\label{thrm:lin::eigenstab::stab}
  Рассмотрим возмущенную задачу: $\ov~Ax = \ov~b$.
  Введём относительную и абсолютную погрешность $A, x, b$:
  \[
    \begin{aligned}
      &ΔA = \ov~A - A, &  &Δx = x - x^*, & &Δb = \ov~b - b \\  
      &δ_A = \frac{\norm{ΔA}}{\norm{A}}, & 
      &δ_x = \frac{\norm{Δx}}{\norm{x*}}, &  
      &δ_b = \frac{\norm{Δb}}{\norm{b}} \\  
      x^* &\text{~--- невозмущенное решение} \span\omit\span
    \end{aligned}
  \]
  Тогда
  \[
    δ_x \leqslant \frac{μ(A)}{1-μ(A)δ_A}\, (δ_A + δ_b)
  \]
\end{thrm}
\begin{prf}
  Раз $x^*$~"--- решение $Ax^* = b$ 
  \[
    \begin{aligned}
      A'x = b' &\iff (A + ΔA) \, (x^* + Δx) = b + Δb \iff (A + ΔA)Δx  = -ΔA x^* + Δb \\
               &\iff \Bigl(I - (-A^{-1} ΔA)\Bigr) \,\frac{Δx}{x^*} = -A^{-1}ΔA +
                 A^{-1}\frac{Δb}{x^*}
    \end{aligned}
  \]
  Из леммы выше 
  \[
    \norm{\frac{Δx}{x^*}} \leqslant \frac{1}{1-\norm{A^{\smash{-1}}}\norm{ΔA}}\, 
    \left( \|A^{-1}\|\norm{ΔA} + \|A^{-1}\|\norm{\frac{Δb}{x^*}}\right)
  \]
  Из невозмущённой системы $\norm{x^*} \geqslant \|A\|^{-1}\norm{b}$, вспомнив определение 
  числа обусловленности осознаем $\|A^{-1}\|\norm{ΔA} = μ(A) \, δ_A$.
  Осталось переписать остальное через $δ$ и получить утверждение теоремы.
\end{prf}

\begin{rem}
  Из этой теоремы можно прикинуть ошибку решения ЛСУ. 
  Будем, как и обещали, использовать обратный анализ ошибок.
  Из-за неточного представления в памяти $δ_A, δ_b \sim ε_M$ (ну никак не меньше),
  так что $δ_x \sim C(s)\, μ(A)\, ε_M$‚ $C(s)$~"--- функция параметров задачи.
\end{rem}
\begin{rem}
  На оценку погрешности ещё влияют индивидуальные особенности методов.
  Например, в методе исключения Гаусса часто накапливается ошибка из-за
  деления на маленькие ведущие элементы.
\end{rem}

\clause{Поиск собственных чисел}${}$\\

Некий полезный набор фактов из линейной алгебры, который совсем не
стоит забывать
\begin{enumerate}
  \item $Au - λu$~"---  уравнение на собственные числа и собственные вектора.
  \item $p_A(t) = \det (A -tI)$~"--- характеристический многочлен.
  \item матрицы можно приводить к ЖНФ
  \item ЖНФ~"--- диагональ из жордановых клеток:
    \[
      J_p(a) = \begin{pmatrix}
        a & 1      & \\
          & \ddots & 1\\
          &        & a\\
      \end{pmatrix}\colon\;  p × p, \qquad p_{J_p(a)}(t) = (a-t)^p 
    \]
  \item алгебраическая кратность собственного числа~"--- кратность его как корня
    характеристического многочлена. Совпадает с размерностью корневого
    подпространства ($V(λ)$).
  \item геометрическая кратность~"--- размерность собственного подпространства
    $(V_λ)$.
  \item геометрическая кратность $\leqslant$ алгебраической, 
    ибо $\dim V_λ \leqslant V(λ)$.
  \item собственные числа самосопряженных операторов вещественные.
  \item из собственных векторов самосопряжённого оператора можно собрать ортогональный
    базис.
\end{enumerate}
\begin{prop}
  Для самосопряжённого положительно определённого оператора
  \[
    \max λ_A = \max \frac {(Au,u)}{(u,u)}, \quad \min λ_A = \min \frac {(Au,u)}{(u,u)}
  \]
\end{prop}
\begin{prf}
  Например, через теорему об условном экстремуме
\end{prf}
\begin{prop}
  $\norm2{A} = \sqrt{\max λ_{A^*A}}$
\end{prop}
\begin{prf}
  Эвклидова норма согласована со скалярным произведением, так что
  \[
    \max \frac{\norm2{Ax}}{\norm2{x}} = \max \frac{(Ax,Ax)}{(x,x)} = \max
    \frac{(A^*Ax,x)}{(x,x)}.
  \]
  А дальше можно глянуть утверждение выше.
\end{prf}
В принципе это сработает для любой нормы, согласованной со скалярным произведением.


Теперь наконец обсудим устойчивость
\begin{exmp}\label{exm:lin::eigenstab::unstab}
  Пусть \[
    A=J_p(a),\quad εB \that (εB)_{ij} = δ_{ip} δ_{j1}, \quad \ov~A = A+εB
  \]
  Оценим ошибку собственного числа 
  \[
    Ax + εBx = λx \iff \left\{\begin{aligned}
        ax_k + x_{k+1} &= λx_k, & k &\in 1\intrng p-1\\
        εx_1 + ax_p &= λx_p
    \end{aligned}\right.
    \so ε x_1 = (λ-a)^{p}x_1
  \]
  В итоге получается, что $λ = a + ε^{1/s}$\note{можно конечно корень из 1 в $\C$
  посчитать, но идея не изменится}

  Пусть $ε = 10^{-16}$ (удвоенная точность). Тогда уже на матрицах
  порядка $15$ ошибка $\sim 0.1$. Грустная оценка получилась.
\end{exmp}

\paragraph{Теорема Бауэра-Файка}
\label{par:lin::bf}

ситуация немного лучше, когда матрицы симметричные. Можно придумать
не такие грустные оценки, как в примере в предыдущем параграфе.

\begin{thrm}\label{thrm:lin::bf}
  Пусть $A$~"--- диагонализуемая матрица, $D^{-1}A D = Λ$. Тогда
  \[
    λ_{A+B} \text{~"--- с.ч. $A+B$}\so
    \exists\, λ_A\that\abs{λ_{A+B} - λ_A} \leqslant μ(D)\norm{B} 
  \]
\end{thrm}
\begin{prf}
  Построим отрицание 
  \[
    \forall\, λ_A\holds\abs{z - λ_A} > μ(D)\norm{B} \so 
    z \text{~"--- не с.ч. $A+B$}
  \]
  и будем его доказывать. Пусть $\abs{z-λ_A} > \norm{B}$. $\gprov$ $A+B-zI$~"--- неособая
  \[
    A - zI + B = D^{-1}\,(Λ-zI + DBD^{-1})\, D = D^{-1} \, (Λ-zI) \, 
    \bigl(I+\underbrace{(Λ-zI)^{-1} D BD^{-1}}_C\bigr)\, D
  \]
  $D$, $(Λ-zI)$ неособые по условию. Воспользуемся леммой об обратимости
  (\ref{lem:lin::eigenstab::idaddinv}). Для этого нужно $\norm{C} < 1$:
  \[
    \norm{C} \leqslant \norm{(Λ - zI)^{\smash{-1}}}
    \underbrace{\|D^{-1}\| \norm{B} \norm{D}}_{μ(D)\norm{B}}
  \]
  Из утверждения в предыдущем параграфе, и отрицания к предположению теоремы 
  \[
    \norm{(Λ - zI)^{\smash{-1}}} = \sqrt{\max_k \abs{λ_k - z}^{-2}} 
    = \sqrt{\frac1{\min_k \abs{λ_k - z}^2}}  < \frac{1}{μ(D)\norm{B}}
  \]
  Как видно, у нас как раз получилось что $\norm{C} < 1$. А тогда и матрица выше обратима.
\end{prf}

\begin{cor}\label{cor:lin::bf::sadj}
  Для самосопряженных матриц
  \[
    λ_{A+B} \text{~"--- с.ч. $A+B$}\so
    \exists\, λ_A\that\abs{λ_{A+B} - λ_A} \leqslant \norm{B} 
  \]
\end{cor}
\begin{prf}
  Для них просто $D$~"---  унитарная, $\norm{D} = \norm{D^{\smash{-1}}} = 1$.
\end{prf}

По сути мы сейчас доказали что собственные числа устойчивы к возмущениям матрицы.
А вот что там с собственными векторами?

\paragraph{Устойчивость собственных векторов при возмущении матрицы}
\label{par:lin::eivstab}

Сразу поясним, какие вообще возникнут проблемы
\begin{exmp}\label{exmp:lin::eivstab::rotnostab}
  Пусть $A_1,A_2$ имеют разные с.ч. и с.в, а
  \[
    C = \begin{cases}I + εA_1, & ε \geqslant 0 \\ I + εA_2, & ε<0 \end{cases}.
  \]
  Тогда, как видно
  \[
    λ_C = \begin{cases}1 + ε λ_1, & ε \geqslant 0 \\ 1 + ε λ_2, & ε<0 \end{cases}, \quad
    u_C = \begin{cases}u_1, & ε \geqslant 0 \\ u_2, & ε<0 \end{cases}
  \]
  и в $u$ никакого $ε$ нету. Направление у них изменяется скачком при проходе через 0.
  А вот с $λ$ всё хорошо.

\end{exmp}
Проблемы, как видно, возникают в окрестности кратных собственных чисел, снятие вырождения
радикально меняет собственные подпространства. Давайте не делать кратных собственных чисел.
Может быть так всё будет хорошо? 

\begin{prop}\label{prop:::peigstab}
  Пусть $(λ_i, u_i)$~"---  собственные числа и векторы $A$, $(μ_i, v_i)$~"--- $A^*$,
  все $λ_i$ разные. Короче говоря, $A$ диагонализуема, но может быть не самосопряжённой.
  Рассмотрим возмущенную задачу на собственные числа и векторы: 
  $(A + ΔA)\,x = (λ+Δλ)\,x$.

  Тогда в линейном приближении\note{а если нет, то надо думать}
  \begin{enumerate}\everymath{\displaystyle}
    \item[\bullet] $p_i = \frac{\norm{u_i}\norm{v_i}}{(u_i, v_i)}$
    \item $\norm{Δλ_i} \leqslant p_i \norm{ΔA}$
    \item $δ{u_i} \leqslant \sum_{k=1}^s \frac{p_k}{\abs{λ_i - λ_k}} \norm{ΔA}$
  \end{enumerate}
\end{prop}

\begin{prf} Пойдём по порядку. 
  \begin{enumerate}\everymath{\displaystyle}
    \item $λ_i = \ov-{μ_i}$
    \item $(u_i, v_k) = 0$ при $i\neq k$
      \[
        \begin{aligned}
          (Au_i, v_k) &= λ_i (u_i, v_k) \\
          (u_i, A^*v_k) &= \ov-{μ_k} (u_i, v_k) \\
        \end{aligned} \so (λ_i - \ov-{μ_k}) \, (u_i, v_k) = 0 \so (λ_i - λ_k) \, (u_i, v_k) = 0
      \]
    \item $(Ax,v_i) = \ov-{\mu_i} (x,v_i) = λ_i (x, v_i)$
    \item $Δ λ_i(u_i, v_i) = (ΔAu_i, v_i)$
      \[
        (\ov~A \ov~u_i,v_i) = (\ov~λ_i\ov~u_i, v_i)  
        \so (ΔAu_i,  v_i) + 
        (A \overbracket[0.5pt]{Δu_i, v_i) = Δ λ_i (u_i, v_i) + λ_i(Δu_i}^{=}, v_i)
      \]
      отсюда уже легко вывести первый пункт.
    \item $(Δu_i, v_k) = (1-δ_{ik}) \, \frac{(ΔA u_i, v_k)}{(λ_i, λ_k)}$, аналогично предыдущему 
      пункту. Здесь выбрали $(Δu_i, v_i) = 0$ пожертвовав нормированностью $v_i$. Всё равно 
      одна лишняя степень свободы была.
    \item $Δu_i = \sum_{k=1}^s γ_k u_k$ \note{$u_i$ образуют базис, раз матрица диагонализуема; 
      корневые подпространства совпадают с собственными}, $γ_k = \frac{(Δu_i, v_k)}{(u_k, v_k)}$
  \item $Δu_i = \sum_{k=1}^s \frac{(ΔA u_i, v_k)}{(λ_i - λ_k)}\, \frac{1}{(u_i, v_k)}$, 
    отсюда очевиден второй
  \end{enumerate}
\end{prf}

\paragraph{Степенной метод}
\label{par:lin::powermethod}

\begin{defn}[Степенной метод]\label{defn:lin::powermethod}
  Пусть $A$~"--- диагонализуемая матрица порядка $s$.
  Построим итерации такого сорта, $x_0$ выбирается случайно.
  \[
    \ov~x_{n+1} = Ax_n, \quad x_{n+1} = \frac{\ov~x_{n+1}}{\ov~x_{n+1}^1} \quad
    \text{ (делим на первую компоненту) }
  \]
  Будем брать $\ov~x_{n+1}^1$ как оценку наибольшего собственного числа $A$ 
\end{defn}

Мотивировка у такого определения понятная~"---  если $x_n$ разложить по собственным векторам
$A$, то через много шагов наибольшее собственное число забьет все остальные.
Однако, нужно аккуратно сформулировать условия сходимости.

\begin{prop}\label{prop:lin::powermethod::conv}
Пусть для собственных чисел $A$ выполнено условие:
\[
  \abs{λ_1} > \abs{λ_2} \geqslant \dotsb \geqslant \abs{λ_s}
\]
Тогда степенной метод сходится к $λ_1$
\end{prop}
\begin{prf}
  Из определения степенного метода
  \[
    x_n = \frac{\ov~x_n}{\ov~x_n^1} = \frac{A x_{n-1}}{\left\{ A x_{n-1} \right\}^1} 
    = \frac{\left(\ov~x_{n-1}^1\right)^{-1}\, A\ov~x_{n-1} }
    {\left(\ov~x_{n-1}^1\right)^{-1}\, \left\{ A \ov~x_{n-1} \right\}^1} 
    = \dotsb = \frac{A^n x_0}{\left\{A^n x_0\right\}^{1}}
  \]
  
  Разложим $x_0$ по собственным векторам $A$, тогда
  \[
    x_0 = \sum_{k=1}^s c_k u_k \so A^n x_0  = c_1 λ_1^n \, u_1 + \sum_{k=2}^s c_k λ^n_k\, u_k 
  \]

  Отсюда переходим к пределу
  \[
    \begin{aligned}
      x_n &=  \frac{A^n x_0}{\left\{A^n x_0\right\}^{1}} = 
      \frac{c_1 λ_1^n \, u_1 + \sum_{k=2}^s c_k λ^n_k\, u_k }{c_1 λ_1^n \, u_1^1 + \sum_{k=2}^s c_k λ^n_k\, u_k^1 } =
      \frac{\frac{u_1}{u_1^1} + \sum_{k=2}^s c'_k \left(\frac{λ_k}{λ_1}\right)^n\, u'_k }
      {1 + \sum_{k=2}^s c'_k \left(\frac{λ_k}{λ_1}\right)^n\, {u'_k}^1 }  \xto{n\to \infty} \frac{u_1}{u_1^1}\\
      \so \ov~x_{n+1}^1 &= \left\{Ax_n\right\}^1 \xto{n\to\infty}  \frac{\{λ_1 u_1\}^1}{u_1^1} = λ_1
    \end{aligned}
  \]
  Так сработает для комплексных $λ$, поскольку 
  \[
    \lim_{x\to \infty}\abs{\left(\frac{λ_k}{λ_1}\right)^n - 0 } = 
    \lim_{x\to ∞}\left( \frac{\abs{λ_k}}{\abs{λ_1}} \right)^n = 0
  \]
\end{prf}

Посмотрим, что будет если нарушить условия утверждения выше
\begin{exmp}
  $λ_1 = λ_2 = λ$: ну это одно и тоже число, так неинтересно
\end{exmp}
\begin{exmp}
  $λ_1 = -λ_2 = λ$
  \[
    \begin{aligned}
      x_{2n} &\to \tfrac{c_1u_1 + c_2u_2}{c_1u_1^1 + c_2 u_2^1}, &
      \ov~x_{2n+1}^1 &\to  λ\tfrac{c_1u^1_1 - c_2u^1_2}{c_1u_1^1 + c_2 u_2^1} \\
      x_{2n+1} &\to \tfrac{c_1u_1 - c_2u_2}{c_1u_1^1 - c_2 u_2^1}, &
      \ov~x_{2n+2}^1 &\to  λ\tfrac{c_1u^1_1 + c_2u^1_2}{c_1u_1^1 + c_2 u_2^1} \\
    \end{aligned}
  \]
  подпоследовательности сходятся к разным числам
\end{exmp}

\begin{exmp}
  $λ_1 = Re^{i θ}$, $λ_2 = Re^{-i θ}$, раз матрица вещественная $c_2 = \ov-{c_1}$, $u_2 =
  \ov-{u_1}$
  \[
    \begin{aligned}
      x_{n} &\to \tfrac{2\Re\left(c_1u_1e^{inθ}\right)}{2\Re\left(c_1u_1^1e^{inθ}\right)}, &
      \ov~x_{n+1}^1 &\to R \tfrac{\Re\left(c_1u_1^1e^{i(n+1)θ}\right)}{\Re\left(c_1u_1^1e^{inθ}\right)}\\
    \end{aligned}
  \]
  кажется это вообще никуда не сходится.
\end{exmp}

Для недиагонализуемых может сходиться, но медленно.
\plholdev{пример с матрицей-производной}

\begin{defn}[Степенной метод со сдвигом]\label{defn:lin::powermethod::powershifted}
  Рассмотрим в степенном методе матрицу $A-tI$ вместо $A$. При этом ищется наиболее удалённое
  по модулю от $t$ собственное число.
\end{defn}

\paragraph{Обратный степенной метод}
\label{par:lin::inversepowermethod}

\begin{defn}[Обратный степенной метод]\label{defn:lin::inversepowermethod}
  Пусть матрица $A$~"---  неособая. Будем применять степенной метод для $A^{-1}$.
  Решим, что то, что нашлось~"---  наименьшее по модулю собственное число.
\end{defn}

\begin{prop}
  Пусть для собственных чисел $A$ выполнено условие:
  \[
    \abs{λ_1} < \abs{λ_2} \leqslant \dotsb \leqslant \abs{λ_s}
  \]
  Тогда обратный степенной метод сходится к $λ_1^{-1}$
\end{prop}
\begin{prf}
  \[
    \begin{aligned}
      A^{-1}x = λ_*x &\iff λx = λ_*^{-1}x = A x \\
        \abs{λ_1} < \abs{λ_2} \leqslant \dotsb \leqslant \abs{λ_s} &\iff
      \abs{λ_1^{-1}} > \abs{λ_2^{-1}} \geqslant \dotsb \geqslant \abs{λ_s^{-1}} 
    \end{aligned}
  \]
\end{prf}

\begin{defn}[Обратный степенной метод со сдвигом]\label{defn:lin::powermethod::invpowershifted}
  Рассмотрим в обратном степенном методе матрицу $A-tI$ вместо $A$.
  Метод при этом будет искать ближайшее к $t$ собственное число
\end{defn}
В частности, можно взять грубую оценку с.ч. и уточнить её таким методом.


\begin{defn}[Обратный степенной метод с переменным
  сдвигом]\label{defn:lin::powermethod::invpowervarshifted}
  Возьмём обратный степенной метод и слегка изменим шаг итерации.
  Помимо махинаций с $\ov~x_{n+1}$,
  \[
    t_{n+1} = t_n + μ_n^{-1}, \qquad μ_n = \ov~x_{n+1}^1
  \]
  Метод при этом будет искать ближайшее к $t_0$ собственное число
\end{defn}
Доказывать что такой алгоритм сходится мы не будем.
Зато можно понять почему он так устроен.
Поскольку $\ov~x_{n+1}$~"--- текущее приближение собственного вектора
\[
  \ov~x_{n+1} = (A-tI)^{-1}x_n \iff (A-tI)\ov~x_{n+1} = x_n \iff \ov~x_{n+1} = (λ_A-t)^{-1}x_n
\]
На каждом шаге$x_n^1 = 1$, так что
\[
  μ = \ov~x_{n+1}^1 = (λ_A - t)^{-1} \iff λ_A = t + μ^{-1}
\]

\paragraph{Двумерные вращения}
\label{par:lin::rot}
Будем рассматривать матрицы самосопряженных операторов. 
На всякий случай, снова приведём набор полезных фактов из
линейной алгебры.
\begin{enumerate}
  \item Унитарные операторы~"--- такие, что сохраняют скалярное произведение:
    \[
      U \that (Ux, Uy) = (x, y)
    \]
  \item $U^*U = 1\so U^{-1} = U^*$
  \item Если $A$~--- самосопряженный, $\exists\, U \that A = U^{-1}ΛU$, $Λ$ тут диагональная.
  \item Произведение унитарных операторов~"---  унитарный оператор
\end{enumerate}

Вернёмся на вещественную прямую.
Матрицы операторов сменили названия
\[
  \begin{aligned}
    \text{унитарные}& &&\to& &\text{ортогональные} \\
    \text{эрмитовы}& &&\to& &\text{симметричные}
  \end{aligned}
\]
Все утверждения выше сохранились.
Запишем ещё пару специфических для $\R^n$ фактов 
\begin{enumerate}[resume]
  \item Существует базис, в котором матрица ортогонального оператора~"--- диагональ из
    блоков такого сорта:
    \begin{description}
      \item[тождество:] \fbox{$1$} 
      \item[отражение:] \fbox{$-1$} 
      \item[вращение:] \fbox{$
          \begin{smallmatrix}
            \cos φ & -\sin φ \\[0.5em]
            \sin φ & \phantom{-}\cos φ
        \end{smallmatrix}$}
    \end{description}
  \item Если вся ортогональная матрица единичная кроме одного
    блока, то
    она простое отражение/вращение.
\end{enumerate}

Простые вращения ещё называются двумерными вращениями.
Все потому, что блоки такого сорта соответствуют двумерным инвариантным подпространстрам.

Будем потихоньку приводить матрицу $A$ к (по возможности) диагональному виду.
Посмотрим, как выглядит один шаг такого приведения
\[
  A \to C = O^T A O, \quad \text{$O$~"--- ортогональная}
\]

Если выполнить все выкладки (для краткости переобозначив $\cos$, $\sin$ за
$c$, $s$), получится
явное выражение для компонент $C$
\begin{equation}\label{eq:lin::rot::step}
  C \that \qquad  
    \begin{aligned}
               c_{ij} &= a_{ij}, & i,j &\neq p,q \\
      c_{pj} = c_{jp} &= c\,a_{pj} + s \,a_{qj}, &\quad j &\neq p,q \\
      c_{qj} = c_{jq} &= -s\,a_{pj} + c \,a_{qj}, &\quad j &\neq p,q \\
               c_{pp} &= a_{pp}\, c^2 + 2\,a_{pq}\, cs + a_{qq}s^2 \\
      c_{pq} = c_{qp} &= (a_{qq} - a_{pp})\,cs + a_{pq}\,(c^2 - s^2) \\
               c_{qq} &= a_{qq}\, c^2 - 2\,a_{pq}\, cs + a_{pp}s^2 \\
  \end{aligned}
\end{equation}

Как можно избавиться от внедиагональных членов:
\begin{enumerate}
  \item $c_{p-1,q} = 0$: вращение Гивенса
\begin{equation*}
  \begin{split}
    c_{p-1,q} &= -s\,a_{p,p-1} + c\, a_{q,p-1},\\
    c &= \cos \varphi, s = \sin \varphi
  \end{split} \so 
  \cos φ = \frac{a_{p-1,p}}{\sqrt{a_{p-1,p}^2 + a_{p-1,q}^2}}; \quad
  \sin φ = \frac{a_{p-1,q}}{\sqrt{a_{p-1,p}^2 + a_{p-1,q}^2}}
\end{equation*}
  \item $c_{p,q} = 0$: вращение Якоби
\begin{equation*}
  \begin{split}
    c_{p,q} &= (a_{qq} - a_{pp})\,cs + a_{pq}\,(c^2 - s^2) \\
    c &= \cos \varphi, s = \sin \varphi
  \end{split} \so 
  \tan φ = \frac{2a_{pq}}{a_{qq}^2 - a_{pp}^2}
\end{equation*}
\end{enumerate}
\paragraph{Лемма о правиле знаков при исключении}
\label{par:lin::signrule}

Вспомним пару фактов из линейной алгебры:
\begin{defn}
  Пусть $B(x, y)$~"--- симметрическая билинейная функция. Тогда функция одного аргумента
  $B(x,x)$ называется квадратичной формой. 
\end{defn}
\begin{enumerate}
  \item $\forall\, B(x,x) \; \exists\, A\that (Ax,x) = B(x,x)$, $A$~"--- самосопряженный.
    Это означает, что можно записать матрицу квадратичной формы и она симметрична.
  \item \emph{Закон инерции}: если привести матрицу квадратичной формы к диагональному виду, то
    количество элементов одного знака не зависит от способа приведения.
\end{enumerate}

Теперь можно сформулировать лемму.
\begin{lem}\label{lem:lin::signrule}
  Пусть $A$~"--- симметричная матрица.  Тогда число ведущих элементов одного
  знака в методе исключения Гаусса для такой матрицы совпадает с числом
  собственных чисел того же знака.
\end{lem}

\begin{prf}
  Рассотрим квадратичную форму $(Ax,x)$. Напишем её в координатах:
  \[
    (Ax,x) = \sum_{ij} a_{ij} x_i x_j = a_{11}x_1^2 + 2\sum_{i} a_{1j}x_1x_j + 
    \sum_{i,j \geqslant 2} a_{ij}x_i.
    x_j
  \]
  Будем приводить её к сумме квадратов стандартным способом (Лежандра).
  \[
    (Ax,x) = a_{11}^{-1} \,\biggl( \underbrace{\sum_{j}a_{1j}x_j}_{ξ_1^2}\biggr)^2
    + \sum_{i,j \geqslant 2} \Bigl(
      \underbrace{a_{ij} - \tfrac{a_{1i}a_{1j}}{a_{11}}}_{a'_{ij}}
    \Bigr)\,x_ix_j
  \]
  Внимательно присмотримся к $a_{ij}'$. Мы вычитаем из элемента строки такой же
  элемент первой строки, поделённый на первый элемент первой строки, умноженный
  на первый элемент данной строки. А это как раз шаг метода исключения Гаусса.

  Теперь приведём $A$ к ЖНФ. Поскольку она симметричная, $J_A$ диагональная. 
  На диагонали стоят собственные числа. 
  Сравнивая их c $a_{jj}$ и припоминая закон инерции, приходим к утверждению
  леммы.
\end{prf}

\paragraph{Метод Гивенса}
\label{par:lin:givens}

Вспомним, как выглядело вращение Гивенса

\begin{equation*}
  \begin{split}
    c_{p-1,q} &= -s\,a_{p,p-1} + c\, a_{q,p-1} = 0,\\
    c &= \cos \varphi, s = \sin \varphi
  \end{split} \so 
  \cos φ = \frac{a_{p-1,p}}{\sqrt{a_{p-1,p}^2 + a_{p-1,q}^2}}; \quad
  \sin φ = \frac{a_{p-1,q}}{\sqrt{a_{p-1,p}^2 + a_{p-1,q}^2}}
\end{equation*}
Будем строить повороты с $(p,q)$ в таком порядке, как на картинке ниже
\begin{center}
\begin{tikzpicture}[>=stealth, 
  thick, black!50, text=black]
  \matrix[row sep = 15, column sep = 20]{
    \node(b1) {$(2,3)$};&\node(p11){$(2,4)$};& \node(p12) {$(2,5)$};&\node(m1) {$\cdots$};&
    \node(e1) {$(2,s)$};\\
                        &\node(b2) {$(3,4)$};&\node(p21) {$(3,5)$};&\node(m2) {$\cdots$};&
    \node(e2) {$(3,s)$};\\
                        &                    &\node(b3) {$(4,5)$}; &\node(m3) {$\cdots$};\\
  };
  \path[->] 
    (b1)  edge (p11)
    (p11) edge (p12)
    (p12) edge (m1)
    (m1)  edge (e1)
    (e1)  edge [out=0, in=180] (b2)
    (b2)  edge (p21)
    (p21) edge (m2)
    (m2)  edge (e2)
    (e2)  edge [out=0,in=180] (b3)
    (b3)  edge (m3)
    ;
\end{tikzpicture}
\end{center}

На каждом шаге метода столбцы/строки с индексами $p$ и $q$ заменяются их линейными комбинациями.
При этом явно зануляется $(p-1,q)$ элемент.
А после предыдущих шагов $\forall j>1$ $(p-j, q)$, $(p-j, p)$ уже нули. 
Либо $p=2$ и выше просто ничего нет. Так что и
линейные комбинации <<верхушек>> столбцов будут нулями, 
и ничего испортиться не сможет. Про область нулей под диагональю можно особо
не думать, она получится автоматически, так как матрица симметричная на каждом
шаге.

После вращений Гивенса матрица стала трёхдиагональной.
\[
  (A - t I) = \begin{pmatrix}
    a_1 - t & b_1     & \\
    b_1     & a_2 - t & \ddots &\\
            &         & \ddots &b_{s-1}\\
            &         & b_{s-1}&a_s - t\\
  \end{pmatrix} 
\]
Введём $p_k$~--- угловые миноры.
порядка $k$ Из формулы разложения определителя по строке получаются рекуррентные
формулы для $p$
\[
  \begin{aligned}
    p_1(t) &= a_1 - t \\
    p_2(t) &= (a_2 - t) p_1(t) - b_1^2 \\
    p_k(t) &= (a_k - t) p_{k-1}(t) - b_{k-1}^2 p_{k-2}(t) \\
  \end{aligned}
\]
Как нетрудно заметить, последовательность $p_k(t)$ для 
фиксированного $t$~"--- это тоже самое что и $a_{kk}$ в лемме в предыдущем
параграфе (\ref{lem:lin::signrule}). Ну ведь правда, после
$k$ шагов метода Гаусса на диагонали вплоть до $k$ строки стоят $1$.
Так что угловой минор просто равен $a_{kk}$. 

Разберёмся теперь как искать собственные числа.
\begin{enumerate}
  \item Как корни характеристического многочлена, $χ(t) = p_s(t)$
  \item Методом бисекции\par
  Про этот пункт придётся написать чуть подробнее. Выберем какие-то 2 начальныx
  приближения $λ$, чтобы искать его между ними. Будем считать число перемен
  знака в последовательности $a_{kk}$.
  Нам нужно добиться чтобы перемена знака была всегда одна. Обычным методом
  половинного деления как раз можно к этому прийти.
\end{enumerate}


\paragraph{Метод Якоби}
\label{par:lin::jacobi}

Вспомним, как выглядело вращение Якоби
\begin{equation*}
  \begin{split}
    c_{p,q} &= (a_{qq} - a_{pp})\,cs + a_{pq}\,(c^2 - s^2) \\
    c &= \cos \varphi, s = \sin \varphi
  \end{split} \so 
  \tan φ = \frac{2a_{pq}}{a_{qq}^2 - a_{pp}^2}
\end{equation*}

Будем пытаться прийти к почти диагональной матрицей. Для этого надо
как-то измерять <<недиагональность>>.
Введём набор величин
\begin{itemize}
  \item $N^2(A) = \sum_{i,k} a_{ik}^{2} = \Tr (A^2)$
  \item $d^2(A) = \sum_{i,k} a_{ii}^{2}$
  \item $t^2(A) = \sum_{i\neq k} a_{ik}^{2} = N^2(A) - d^2(A)$
\end{itemize}

\begin{prop}\label{prop:lin::jacobi::nondiagest}
  После одного двумерного вращения ($C = O_{pq}^{T}AO_{pq}$)
  \[
    t^2(C) = t^2(A) - 2 a_{p,q} + 2 c_{p,q}^2
  \]
\end{prop}
\begin{prf}Пойдем по порядку
  \begin{enumerate}
    \item $N^2(C) = N^2(A)$, поскольку $C^2 = O^TAO\,O^TAO = O^{T}AO$, а след 
      подобных матриц совпадает.
    \item $t^2(C) = N^2(C) - d^2(C) = t^2(A) + d^2(A) - d^2(C)$
    \item $d^2(A) - d^2(C) = a_{p,p}^2 + a_{q,q}^2 - c_{p,p}^2 - c_{q,q}^2$, просто
      все остальные элементы на диагонали не поменялись.
    \item $a_{p,p}^2 + a_{q,q}^2 + 2a_{p,q}^2 = c_{p,p}^2 + c_{q,q}^2 + 2c_{p,q}^2$.

      Это можно либо явно проверить из формулы~\eqref{eq:lin::rot::step}, либо
      вспомнить что вращения квадраты норм матриц не изменяют, а эти 4 элемента
      преобразуются независимо от других. Разве что мы норму оператора не так 
      определяли.
  \end{enumerate}
\end{prf}
Метод Якоби как раз зануляет $c_{pq}^2$ на шаге, оптимально уменьшая таким образом
$t^2$.
Разберёмся как выбирать здесь $p$ и $q$. 

\begin{enumerate}
  \item Классический метод Якоби: 
    $p,q\that \abs{a_{p,q}} = \max\limits_{i\neq k}\abs{a_{i,k}}$.
    
    Оценим, как быстро он сходится
    \[
      a_{p,q}^2\, \tfrac{s\,(s-1)}2 \geqslant \sum_{i,k} a_{i,k}^2 \so
      t^2(C) \leqslant \left(1 - \tfrac{2}{s(s-1)}\right)\, t^2(A) 
    \]
    Неплохо, но поиск максимума $\sim O(s^2)$, а сам метод
    Якоби $\sim O(s)$. Подумаем как можно улучшить.
  \item Циклический метод Якоби: просто проходим по всем наддиагональным
    элементам много раз. 
  \item Циклический метод Якоби c барьером: выбираем $ε_i > 0$ и зануляем всё
    что больше него. Потом выбираем $ε_{i+1} < ε_{i}$ и повторяем.
\end{enumerate}

Разберёмся, как искать собственные числа и собственные векторы.
\begin{enumerate}
  \item С $λ$ всё просто~"---  они на диагонали матрицы. Корректность следует
    из теоремы Бауэра-Файка (\ref{thrm:lin::bf}), просто вычтем внедиагональные
    члены
  \item в качестве собственных векторов можно просто взять строки матрицы
    произведения всех двумерных вращений.
    
    Это сработает, поскольку собственные векторы диагональной
    формы~"--- $e_k$,
    \[
      λ_k e_k = Λ e_k = OA O^{T} e_k \iff A\,O^{T}e_k = λ_k O^{T}e_k,
    \]
    а $O^Te_k$ как раз $k$-ая строчка $O$.
\end{enumerate}

\paragraph{Две леммы о факторизации матрицы}
\label{par:lin::factor}

\begin{lem}\label{lem:lin::factor::lr}
  Пусть $A$~"--- неособая матрица с ненулевыми диагональными минорами
  Тогда
  \[
    \exists!\, L,R \that A = LR, \quad
    L = \begin{pmatrix}
      1 &  & \\
      \bullet  & \ddots  & \\
      \bullet &\bullet &1\\
    \end{pmatrix} , \qquad
    R = \begin{pmatrix}
      \bullet & \bullet & \bullet\\
        & \ddots  & \bullet\\
        &         &\bullet\\
    \end{pmatrix}.
  \]
  То есть раскладывается на произведение верхней/нижней треугольной.
\end{lem}
\begin{prf}
  Эта теорема~"--- матричная запись метода Гаусса. 
  Запишем явное выражение для первого шага
  \[
    a_{ij} = a_{ij} - \frac{a_{i1}a_{1j}}{a_{11}}, \quad i = 2, \dotsc, s
  \]
  Посмотрим на эту формулу как на преобразование $j$го столбца. Тогда матрица 
  такого преобразования имеет вид
  \[
    \begin{pmatrix}
      1 &   & & & \\
      \bullet  & 1 & & \\
      \vdots  & 0 & 1& \\
      \bullet  & 0 & 0& 1\\
    \end{pmatrix}
  \]
  понятно, что в случае $k$-го шага будет просто $k$-й столбик.
  Если мы будем перемножать такие столбики, они просто будут пристраваться рядом.
  Ну в самом деле, умножим нижнетреугольную  матрицу на такой столбик.
  $c_{ij} = \sum_{p}a_{ip}b_{pj}$, а при всех $j \neq k$ вместо $b_{pj}$ просто
  такой же член от единичной матрицы. А сохранение треугольности следует из
  треугольности $a_{ip}$.
  
  Чтобы убедиться в единственности, можно рассмотреть матричное равенство
  построчно. А у последовательного 
  набора этих равенств получается всего одно решение.
\end{prf}


\begin{lem}\label{lem:lin::factor::qr}
  Пусть $A$~"--- неособая матрица.
  Тогда
  \[
    \exists\, Q,R \that A = QR
  \]
  Здесь $R$ как в лемме выше, а $Q$~--- ортогональная.
\end{lem}
\begin{prf}
  Прогоним процесс ортогонализации Грамма-Шмидта для строчек $A$, 
  строчки $Q$ это полученный ортогональный базис. 
  При этом $Q = LA$, только на диагонали не обязательно $1$.
  А $L^{-1}$ уже будет верхнетреугольной.
\end{prf}

\paragraph{Теорема о сходимости итерированных подпространств}
\label{par:lin::iterspaceconv}

Вспомним про степенной метод из \ref{par:lin::powermethod}.
У него был недостаток, он не умел искать больше одного собственного числа.
Но мы и итерировали всего один вектор. Давайте обобщим.

\begin{defn}\label{defn:lin::iterspaceconv::iterbas}
  Пусть $A$~"--- диагонализуемая неособая матрица, 
  $\{x_{j}\}$~"---  базис в $\R^s$, 
  \begin{enumerate}
    \item $\bigl\{x_{j}^{(n)}\bigr\}=\bigl\{A^nu_j\bigr\}$~"--- 
      $n$-ный итерированный базис
    \item $L_s^{(n)} = \left\langle A^n x_1,\dotsc,A^n x_s\right\rangle$~"--- 
      $n$-ное итерированое подпространство.
  \end{enumerate}
\end{defn}
\begin{rem}
  Можно итерировать не весь базис, а, например, только $k$ векторов из $s$.
  Помимо добавления эпитетов, характеризующих размерность, 
  вводят $U_k=\left\langle u_1, \dotsc, u_k\right\rangle$~"---  $k$-мерное
  старшее собственное подпространство, а $\{u_j\}$~"--- базис из собственных
  векторов $A$.
\end{rem}

\begin{defn}\label{defn:lin::iterspaceconv::iterspconv}
  Говорят, что $P^{(n)} \to P$, если в $P^{(n)}$ существует базис, сходящийся к
  базису $P$.
\end{defn}

\begin{thrm}\label{thrm:lin::iterspaceconv::conv}
  Пусть $A$~"--- диагонализуемая неособая матрица, 
  \[
    \abs{λ_1} > \abs{λ_2} > \cdots > \abs{λ_s} > 0 \qquad\text{(все разные)}
  \]
  Тогда $L_k^{(n)} \to U_k$.
\end{thrm}
\begin{prf}
  Соорудим базис в $L_k^{(n)}$ который будем сходится к базису $U_k$.
  Пусть ${x_j}^k$~"---  исходный базис,\note{этого условия у нас не было, но
  без него не доказать невырожденность $\ov~C$.}
  \[
    \langle x_1, \dotsc, x_k \rangle \supset \langle u_1, \dotsc, u_k \rangle.
  \]
  Разложим его по $u_j$ и посмотрим на итерированный
  \[
    x^{(n)}_i = \sum_{\ell=1}^k c_{i\ell} \, λ_\ell^n u_\ell 
    + \sum_{\ell=k+1}^s c_{i\ell} \,λ_\ell^n u_\ell
  \]
  Домножим обе части на $D = \left(\ov~C\right)^{-1}$, $\ov~C$~"--- квадратный
  кусок $C$ размерами $k \times k$ из перых коэффициентов.

Рассмотрим $\ov~Z_{m}^{(n)} = \sum_{i=1}^k d_{m,i}\,x_{i}^{(n)}$, они явно базис $L_k^{(n)}$
в силу невырожденности $D$.
\[
\ov~Z^{(n)}_{m} = 
\sum_{\ell,i=1}^k \underbrace{d_{m,i}\,c_{i,\ell}}_{δ_{m\ell}} \,λ_{\ell}^n u_{\ell} + 
  \sum_{\ell=k+1,i=1}^{s,k} d_{m,i}\,c_{i,\ell} \,λ_{\ell}^n u_{\ell}
\]
Теперь поделим: $Z_{m}^{(n)} = \ov~Z_{m}^n \, λ_{m}^{-n}$, они всё ещё базис
$L_k^{(n)}$.
\[
  Z_{m}^{(n)} = u_{m} + \sum_{\ell \geqslant k+1,i} d_{m,i}\,c_{i,\ell} 
  \,\left(\frac{λ_{\ell}}{λ_m}\right)^n u_{\ell} \to u_m
\]
\end{prf}

\begin{defn}[Ступенчатый базис]\label{defn:lin::iterspaceconv::stairbasis}
  \begin{align}
    e_1 &= (1, x_{12}, \dotsc ) \\
    e_k &= (0, \dotsc, 0,  1, x_{k,k+1}, \dotsc ) \\
  \end{align}
\end{defn}

\begin{prop}
  Обычно базис пространства приводится к ступенчатому.
\end{prop}
\begin{prf}
  метод Гаусса
\end{prf}

можно ещё рассматривать, например, $e_1, \dotsc, e_k$ как базис $L_k$, нам
ведь неважно что там в следущих компонентах происходит.

\begin{thrm}\label{thrm:lin::iterspaceconv::stairbasis}
  Cтупенчатый базис $L_k^{(n)}$ сходится к базису $U_k$ при грамотно заданных
  условиях невырожденности.
\end{thrm}
% \begin{prf}
%   Сузим на подпространство и разложим по $Z_{m}^{(n)}$. Невероятно увлекательно
% \end{prf}

\paragraph{Треугольно-степенной метод и его сходимость}
\label{par:lin::trpowermethod}

\begin{defn}\label{defn:lin::trpowermethod}
  Рассмотрим $A$ со стандартными условиями на собственные вектора, 
  произвольную невырожденную $P_0$.
  Шаг итерации выглядит так: 
  \[
    AP_{n} = P_{n+1}R_{n+1},
  \]где $P_{n+1}$ нижнетреугольная, а $R_{n+1}$ верхнетреугольная.
\end{defn}

\begin{thrm}\label{thrm:lin::trpowermethod::conv}
  При стандартных предположениях и неравенстве нулю диагональных
  миноров $A$ на $λ_A$, $P_k, R_k$ сходятся к $P, R$.
  При этом на диагонали $R$ оказываются собственные числа.
\end{thrm}

\begin{prf}
  \begin{enumerate}
    \item Первые $k$ столбцов $P_n$ образуют ступенчатый базис $L_k^{(n)}$
      \begin{enumerate}
        \item $AP_n$ переводит его снова в базис, так как его можно через $Z_{m}^{(n)}$
          выразить.
        \item $LR$-факторизация выражает строчку $L$ через предыдущие.
      \end{enumerate}
    \item Ступенчатый базис сходится к базису $U_k$
    \item $R_{n} = P_{n}^{-1}AP_{n-1} \xto{n\to\infty} {P}^T A P$, а у подобных матриц
      собственные числа совпадают.
  \end{enumerate}
\end{prf}
скорость сходимости здесь степенная, что видно из теоремы в
\ref{par:lin::iterspaceconv}.

\paragraph{Ортогонально-степенной метод}
\begin{defn}\label{defn:lin::iterspaceconv}
  Рассмотрим $A$ со стандартными условиями на собственные вектора, 
  произвольную невырожденную $С_0$.
  Шаг итерации выглядит так: 
  \[
    AC_{k} = C_{n+1}R_{n+1},
  \]где $C_{n+1}$ ортогональная, а $R$ верхнетреугольная.
\end{defn}

\begin{defn}[Сходимость по форме]\label{defn:lin::trpowermethod::formconv}
  Пусть $B$~"--- блочная треугольная матрица. Тогда говорят, что 
  $A_k$ сходится по форме к $B$, если все элементы ниже квазидиагонали сходятся
  к $0$. А что на диагонали и выше нас не интересует, главное чтобы хоть куда-то
  сходилось.
\end{defn}


\begin{thrm}\label{thrm:lin::trpowermethod::conv}
  При стандартных предположениях на $λ_A$ и неравенстве нулю диагональных
  миноров $A$, $C_n^*AC_n$ по форме сходится к $\hat A$, которая верхнетреугольная.
  При этом на диагонали $\hat A$ оказываются собственные числа.
\end{thrm}

\begin{prf}
  Для сходимости по форме нужно просто чтобы вся поддиагональ сходилась к $0$.
  Т.е для $j>k$
  \[
    \{C_n^*AC_n\}_{j,k} \to 0 \iff \left(AC_n^{(k)}, C_n^{(j)}\right) \to 0
  \]
  Первые $k$ строк $C$ образуют ортогональный базис $L_k^{(n)}$.
  
  Вспомним доказательство теоремы про итерируемые подпространства и скажем что
  $x_i$ оттуда это $C^{(i)}$ сейчас. Тогда, $C^{(i)}$ раскладываются
  по $Z^m$ $+$ ещё какие-то члены порядка 
  $O \left( \abs{\frac{{λ_{k+1}}}{λ_k}}^n \right)$.

  При умножении на $A$ разложение по $Z_m$ не испортилось, 
  так что и $AC_n^{(i)}$ примерно в $L_{k}^{(n+1)}$.
  Тогда, из ортогональности всех столбцов $C_n$
  \[
    \left(AC_n^{(k)}, C_n^{(j)}\right) =
    O \left( \abs{\frac{{λ_{k+1}}}{λ_k}}^n \right)\xto{\to\infty} 0
  \]
\end{prf}

\paragraph{$LR$-алгоритм. Практическая реализация}
\begin{defn}\label{defn:lin::iterspaceconv}
  Рассмотрим $A$ со стандартными условиями на собственные вектора, 
  произвольную невырожденную $С_0$.
  Шаг итерации выглядит так: 
  \[
    \begin{aligned}
      A &= L_{1}R_{1}, \\
      R_n L_n &= L_{n+1}R_{n+1}, \\
    \end{aligned}
  \]где $L_{n+1}$ нижнетреугольная, а $R$ верхнетреугольная.
\end{defn}

Нетрудно заметить, что есть связь этого алгоритма со треугольным степенным
\[
  \begin{aligned}
    P_0 &= I \\
    P_n &= L_{1} \dotsm L_n \\
  \end{aligned}
\]
Подставим 
\[
  \begin{aligned}
    A P_0 &= P_1 R_1         & &\to& A I &= L_1 R_1 \\
    A P_n &= P_{n+1} R_{n+1} & &\to& A\, L_1 \dotsm L_n &= L_1 \dotsm L_{n+1} R_{n+1}
  \end{aligned}
\]
Разберёмся c $A\, L_1 \dotsm L_n$
\[
  A = L_1 R_1 = L_1 L_2 \, R_2 \, L_1^{-1} = \dotsb
  = L_1 \dotsm L_n \, R_n  \, L_1^{-1} \dotsm L_{n-1}^{-1}
\]
А подставив такого крокодила как раз получаем $R_nL_n = L_{n+1}R_{n+1}$
Поскольку мы свели метод к предущему, доказывать сходимость уже не нужно.

Приведём пару фактов, нужных для расчетов
\begin{enumerate}
  \item $LR$ факторизация занимает $O(s^3)$ времени. Это очень больно.
    Даже классический метод Якоби на шаге делает $O(s^2)$ работы.
  \item Трехдиагональную матрицу можно факторизовать (прогонкой:) за
    $O(s)$ на двудиагональные.
  \item Если $A$ не трехдиагональная, но симметричная, вращения Гивенса нам помогут.
  \item Можно привести $A$ к трёхдиагональному виду даже если она несимметричная.
  \item Можно ускорить сходимость взяв сдвиг по Реллею
    \[
      R_n L_n - t_n I = L_{n+1} R_{n+1}, \qquad t_n = r_{ss}
    \]
    (последний элемент).
\end{enumerate}

\paragraph{$QR$-алгоритм. Практическая реализация}

\begin{defn}\label{defn:lin::iterspaceconv}
  Рассмотрим $A$ со стандартными условиями на собственные вектора, 
  произвольную невырожденную $С_0$.
  Шаг итерации выглядит так: 
  \[
    \begin{aligned}
      A &= Q_{1}R_{1}, \\
      R_n Q_n &= Q_{n+1}R_{n+1}, \\
    \end{aligned}
  \]где $C_{n+1}$ ортогональная, а $R$ верхнетреугольная.
\end{defn}

сходимость доказывается аналогично $LR$.

Приведём пару фактов, нужных для расчетов
\begin{enumerate}
  \item $QR$-факторизация занимает $O(s^3)$ времени. И это всё ещё больно.
  \item $QR$-факторизация сохраняет трёхдиагональность $A$, но только для
    симметричных
  \item Чтобы ускорить жизнь до $O(s^2)$  привести $Q$
    к форме Хессенберга вращениями Гивенса.
    В такой форме просто есть ещё одна диагональ по сравнению с
    верхнетреугольной.
  \item Можно ускорить сходимость, соорудив сдвиг.
    \[
      R_n Q_n - t_n I = Q_{n+1} R_{n+1}
    \]
    \begin{enumerate}
      \item По Реллею: $\{R_nQ_n\}_{s,s}$
      \item По Уилкинсону: собтвенные числа матрицы $2 × 2$ $\{R_nQ_n\}^{s,s}_{s-1,s-1}$
    \end{enumerate}
\end{enumerate}

\end{document}
% vim:wrapmargin=3


\chapter{Интегральные уравнения}
\label{chap:inteq}
\documentclass{trlnotes}
\setlayout{hardcopy}
\usepackage{silence}
\WarningFilter{latex}{Reference}
\graphicspath{{../../img/}}

\begin{document}
    \paragraph{Интегральное уравнение II рода, метод замены ядра на вырожденное}

    \begin{de}
        \ti{Интегральным уравнением Фредгольма II рода} называется уравнение вида
        \begin{equation} \label{eq:fred2}
            \varphi(x) = f(x) + \mu \int\limits_a^b K(x, \, t) \varphi(t) \, \del t.
        \end{equation}
        Функция $K$~--- его \ti{ядро}, а $\mu$~--- \ti{характеристическое число}.\footnote{Кажется, иногда в определении полагают $\mu = 1$, но всегда ведь можно внести его в ядро.}
    \end{de}

    Обозначим через $K$ (хм, да, вольность) оператор 
    \[
        \varphi(t) \mapsto \int\limits_a^b K(x, \, t) \varphi(t) \, \del t.
    \]
    Ясно, что он компактен. Уравнение теперь примет вид
    \[
        (I - \mu K)\varphi = f.
    \]
    Оператор $T = I - \mu K$, конечно, фредгольмов.

    \begin{st}
        Сопряжённый в $L^2\big([a, \, b]\big)$ оператор к $K$ выражается следующим образом:
        \[
            K^* \varphi(x) = \int\limits_a^b \ov{-}{K(t, \, x)} \varphi(t) \, \del t.
        \]
        \begin{proof}
            Прямым вычислением (ну, там внутри ещё теорема Фубини) проверяется, что 
            \[
                \langle K \varphi, \, \psi \rangle = \langle \varphi, \, K^{*} \psi \rangle.
            \]
        \end{proof}
    \end{st}

    \begin{rem}
        У ядра меняются местами аргументы и оно сопрягается~--- точно так же, как транспонирование вместе с комплексным сопряжением дают матрицу сопряжённого оператора в конечномерном случае!
    \end{rem}

    Сформулируем альтернативу Фредгольма \ref{thm:fred-alt} для такого уравнения:

    \begin{st}
        $\hphantom{.}$
        \begin{enumerate}
            \item Уравнение $T\varphi = f$ разрешимо однозначно тогда и только тогда, когда $\mu^{-1}$~--- не собственное число оператора $K$.
            \item В противном случае уравнение $T \varphi = f$ разрешимо тогда и только тогда, когда функция $f$ ортогональна всем собственным векторам оператора $K^*$, соответствующим числу $\ov{-}{\mu}^{-1}$.
            \item $\mu^{-1}$ и $\ov{-}{\mu}^{-1}$~--- собственные числа операторов $K$ и $K^*$ соответственно одинаковой конечной кратности.
        \end{enumerate}
    \end{st}

    \begin{rem}
        Для симметричного ядра (т.е. когда $K = K^*$) то же самое несложно доказать, используя разложение по собственному базису оператора $K$ (которое есть по теореме Гильберта-Шмидта \ref{thm:hilb-sch}). Так можно быстро понять, что если $\mu^{-1}$~--- собственное число $K$, то решений либо нет, либо их бесконечно много.
    \end{rem}

    Рассмотрим уравнение \ref{eq:fred2} с вырожденным ядром
    \[
        K(x, \, t) = \sum\limits_{i = 1}^n \alpha_i(x) \beta_i(t).
    \]
    Функции $\alpha_i$ и $\beta_i$ можно считать ЛНЗ: если это не так, нетрудно выразить одну из них через другие и избавиться от неё. Подставляя ядро в уравнение \ref{eq:fred2}, получим

    \begin{equation}\label{eq:deg-ker-repr}
        \varphi(x) = f(x) + \sum\limits_{j = 1}^n A_j \alpha_i(x), \text{ где } A_j = \mu \int\limits_a^b \beta_j(t) \varphi(t) \, \del t.
    \end{equation}

    Это представление для функции $\varphi$ теперь подставим в исходное уравнение:

    \[
        f(x) + \sum\limits_{i = 1}^n A_i \alpha_i(x) = f(x) + \mu \int\limits_a^b \sum\limits_{i = 1}^n \alpha_i(x) \beta_i(t) \left(f(t) + \sum\limits_{j = 1}^n A_j \alpha_j(t) \right) \, \del t
    \]
    Чтобы переписать это покороче, введём обозначения
    \[
        \beta_{ij} = \int\limits_a^b \beta_i(t) \alpha_j(t) \, \del t, \quad f_i = \int\limits_a^b f(t) \beta_i(t) \, \del t.
    \]
    и получим
    \[
        \sum\limits_{i = 1}^n A_i \alpha_i(x) = \mu \sum\limits_{i = 1}^n \left(f_i + \sum\limits_{j = 1}^n \beta_{ij} A_j \right) \alpha_i(x).
    \] 
    Поскольку $\alpha_i$ линейно независимы, коэффициенты при них слева и справа должны быть равны. Записав эти равенства, мы приходим к системе линейных уравнений
    \[
        A_i = \mu f_i + \mu \sum\limits_{j = 1}^n \beta_{ij} A_j .
    \]
    В векторном виде она будет выглядеть так:
    \[
        A = \mu (\beta A + f),
    \]
    где $A$ и $f$~--- векторы, $\beta$~--- матрица, а $\mu$ всё ещё число.

    Эта система решается так:
    \[
        (I - \mu \beta) A = \mu f \so \boxed{A = \mu (I - \mu \beta)^{-1} f}\,, \text{ если } \det (I - \mu \beta) \neq 0.
    \]

    Пусть $\Delta = \det(I - \mu \beta)$ и $\Delta_{ij}$~--- алгебраическое дополнение элемента $\delta_{ij} - \mu \beta_{ij}$. Тогда можно записать явную формулу для $A$\footnote{Это просто формула для обратной матрицы через алгебраические дополнения.}:
    \[
        A_{i} = \dfrac{\mu}{\Delta} \sum\limits_{j = 1}^n \Delta_{ji} f_j
    \]
    Подставляя теперь найденные $A_i$ в \ref{eq:deg-ker-repr}, найдём, что
    \[
        \varphi(x) = f(x) + \lambda \int\limits_a^b \Gamma(x, \, t) f(t) \, \del t,
    \]
    где \ti{резольвента} $\Gamma$ имеет вид
    \[
        \Gamma(x, \, t) = \dfrac{1}{\Delta} \sum\limits_{i, \, j = 1}^n \Delta_{ji} \alpha_i(x) \beta_j(s).
    \]

    Трудная задача~--- приблизить произвольное ядро вырожденным. Есть несколько способов:

    \begin{enumerate}
        \item Разложить ядро в ряд Тейлора.
        \item Интерполировать ядро.
        \item Разложить ядро по ортогональной системе функций.
    \end{enumerate}


    Подробнее про них можно прочитать в книге \cite{comp-krilov-2}.

    Заменяя ядро на вырожденное, мы надеемся, что и решения тоже изменятся не сильно. Надо бы это обосновать (хотя бы как-то). Пусть есть уравнение
    \[
        Au = f, \quad A = I - K
    \]
    и приближающее его уравнение
    \[
        A_n u_n = f, \quad A_n = I - K_n. 
    \]
    Нетрудно видеть, что
    \[
        u - u_n = (A^{-1} - A_n^{-1})f \so \|u - u_n\| \leqslant \big\|A^{-1} - A_n^{-1}\big\| \cdot \|f\|.
    \]
    Поэтому интересно оценить норму разности обратных операторов. Займёмся этим.

    \begin{st}\label{st:close-zero-inv}
        Пусть $P$~--- ограниченный оператор, $\|P\| < 1$. Тогда оператор $I - P$ обратим, причём
        \[
            (I - P)^{-1} = \sum\limits_{i = 1}^{\infty} P^n,
        \]
        где сходимость~--- по операторной норме.
    \end{st}

    \begin{st} \label{st:close-inv}
        Пусть $P$ и $H$~--- ограниченные операторы, $P$ обратим, а $\|H\| < \|P^{-1}\|^{-1}$. Тогда элемент $P - H$ обратим, причём
        \[
            \big\| \, (P - H)^{-1}\big\|  \leqslant \dfrac{\|P^{-1}\|}{1 - \|H\| \, \|P^{-1}\|}. 
        \]
        и
        \[
            \big\| \, (P - H)^{-1} - P^{-1}\big\| \leqslant \dfrac{\|H\| \, \|P^{-1}\|^2}{1 - \|H\| \, \|P^{-1}\|}.
        \]
        \begin{proof}
            Позволим себе иногда использовать дроби и $1$ вместо $I$, как если бы операторы были числами. Не составит труда переписать всё через обратные!

            Заметим, что первое из двух утверждений теоремы для $P = I$ следует из \ref{st:close-zero-inv}:
            \begin{equation}\label{eq:close-zero-inv}
                \big\| \, (I - H)^{-1}\big\| = \left \| \, \sum\limits_{i = 1}^{\infty} H^n \, \right\| \leqslant \sum\limits_{i = 1}^{\infty} \|H\|^n = \dfrac{1}{1 - \|H\|}.
            \end{equation}

            Далее,
            \[
                \left\|\, \dfrac{1}{P - H} \,\right \| = \left\|\, P^{-1} \dfrac{1}{1 - P^{-1}H} \,\right \| \leqslant \|P^{-1}\| \cdot \left\|\,\dfrac{1}{1 - P^{-1}H} \,\right \| \leqslant \dfrac{\|P^{-1}\|}{1 - \|P^{-1}H\|} \leqslant \dfrac{\|P^{-1}\|}{1 - \|H\| \, \|P^{-1}\|}.
            \]
            В предпоследнем переходе используется соотношение \ref{eq:close-zero-inv}, где $H \to P^{-1}H$.

            Наконец,
            \[
                \left\|\, \dfrac{1}{P - H} - \dfrac{1}{P}\,\right \| = \left\|\, \dfrac{1}{P}\left(\dfrac{1}{1 - P^{-1}H} - 1 \right)\right \| =  \left\|\, \dfrac{1}{P} \, \dfrac{P^{-1}H}{1 - P^{-1}H} \right \| \leqslant \dfrac{\|H\| \, \|P^{-1}\|^2}{1 - \|H\| \, \|P^{-1}\|}.
            \]
        \end{proof}
    \end{st}

    Отсюда сразу же следует утверждение

    \begin{st}\label{st:dif-inv}
        При достаточно больших $n$
        \[
            \big\|A^{-1} - A_n^{-1}\big\| \leqslant \dfrac{\rho \, \|A^{-1}\|^2}{1 - \rho \, \|A^{-1}\|} \text{ и } \big\|A^{-1} - A_n^{-1}\big\| \leqslant \dfrac{\rho \, \|A_n^{-1}\|^2}{1 - \rho \, \|A_n^{-1}\|},
        \]
        где $\rho = \|A - A_n\| = \|K - K_n\|$.
    \end{st}

    \begin{rem}
        Рассмотрим теперь задачу с симметричным ядром (т.е. с самосопряжённым $K$). В ней есть ортонормированный собственный базис $\alpha_i$, поэтому
        \[
            u = \sum\limits_{i = 1}^{\infty} \langle u, \, \alpha_i \rangle \alpha_i \so Ku = \sum\limits_{i = 1}^{\infty} \langle u, \, \alpha_i \rangle \lambda_i \alpha_i,
        \]
        где $\lambda_i$~--- соответствующее собственное число. Расположим $\lambda_i$ в порядке убывания модуля и положим
        \[
            K_n u = \sum\limits_{i = 1}^{n} \langle u, \, \alpha_i \rangle \lambda_i \alpha_i
        \]
        Это интегральный оператор с вырожденным ядром
        \begin{equation}\label{eq:eig-deg-ker}
            K_n(x, \, t) = \sum\limits_{i = 1}^n \lambda_i \alpha_i(x) \ov{-}{\alpha_i(t)}.
        \end{equation}

        %unsure
        Можно доказать, что он является лучшей аппроксимацией ранга $n$ для оператора $K$ по операторной $L^2$-норме.
        %unsure

        Посмотрим на разность:
        \[
            (K - K_n)u = \sum\limits_{i = n+1}^{\infty} \langle u, \, \alpha_i \rangle \lambda_i \alpha_i.
        \]
        Найдём её норму:
        \[
            \big\|(K - K_n)u \,\big\|^2 = \sum\limits_{i = n + 1}^{\infty} |u_i|^2 \, |\lambda_i|^2, \quad u_i = \langle u, \, \alpha_i \rangle.
        \]
        При этом
        \[
            \|K - K_n\| = \sup \dfrac{\big\|(K - K_n)u \,\big\|}{\|u\|},
        \]
        и
        \[
            \dfrac{\big\|(K - K_n)u \,\big\|^2}{\|u\|^2} = \dfrac{\sum\limits_{i = n + 1}^{\infty} |u_i|^2 \, |\lambda_i|^2}{\sum\limits_{i = n + 1}^{\infty} |u_i|^2} \leqslant \dfrac{\sum\limits_{i = n + 1}^{\infty} |u_i|^2 \, |\lambda_{n + 1}|^2}{\sum\limits_{i = n + 1}^{\infty} |u_i|^2} = |\lambda_{n+1}|^2.
        \]
        С другой стороны, эта оценка достигается, когда $u$~--- собственный вектор числа $\lambda_{n+1}$. Поэтому
        \[
            \boxed{\|K - K_n\| = |\lambda_{n + 1}|} \, .
        \]
        Отсюда и из утверждения \ref{st:dif-inv} ясно: чем быстрее убывают собственные числа, тем лучше наша оценка! 
        %unsure
        Из уравнения \ref{eq:eig-deg-ker} видно, что собственные числа~--- что-то вроде коэффициентов в ряде Фурье по собственным функциям для ядра. Видимо, поэтому скорость их убывания возрастает, если ядро становится более гладким... А ядра гладкие не всегда.
        %unsure
    \end{rem}

    \begin{rem}
        Есть способ сгладить ядро. Надо в уравнение \ref{eq:fred2} подставить
        \[
            \varphi(t) = f(t) + \mu\int\limits_a^b K(t, \, \xi) \varphi(\xi) \, \del \xi.
        \]
        Получится уравнение
        \[
            \varphi(x) = f_2(x) + \mu\int\limits_a^b  K_2(x, \xi) \varphi(\xi) \, \delta \xi,
        \]
        где
        \[
            f_2(x) = f(x) + \mu \int\limits_a^b K(x, \, t) f(t) \, \del t, \quad K_2(x, \, \xi) = \mu \int\limits_a^b K(x, \, t) K(t, \, \xi) \, \del t. 
        \]
        У $K_2$ с гладкостью получше, но его надо считать.
    \end{rem}

    \paragraph{Метод квадратур для интегрального уравнения}

    Идея заключается в том, чтобы в уравнении
    \[
        u(x) = f(x) + \int\limits_a^b K(x, \, t) u(t)\, \del t
    \]
    заменить интегрирование на вычисление по какой-нибудь квадратурной формуле:
    \[
        \int\limits_a^b u (x) \, \del x = \sum\limits_{k = 1}^n A_k u(x_k) + R.
    \]
    Получится
    \[
        u(x) = f(x) + \sum\limits_{k = 1}^n A_k K(x, \, x_k) u(x_k) + R.
    \]
    Пусть $\tilde{u}$~--- решение этого уравнения с отброшенным $R$, $u_k = \tilde{u}(x_k)$, $f_k = f(x_k)$ и $K_{ik} = K(x_i, \, x_k)$.
    Получаем систему линейных уравнений
    \[
        u_i = f_i + \sum\limits_{k = 1}^n A_k K_{ik} u_k.
    \]
    Её можно решить обычными методами; зная $u_k$, можно оценить $u(x)$ в любой точке:
    \[
        u(x) = f(x) + \sum\limits_{k = 1}^n A_k K(x, \, x_k) u_k.
    \]

    Попробуем оценить погрешность результата. Для многих стандартных квадратурных методов верна формула
    \[
        R[\theta] = \delta(n) \max |\theta^{(m)}(x)|.
    \]
    Нас интересует $R\big[K(x, \, t)u(t)\big]$ при фиксированном $x$. $m$-е производные функции $K(x, \, t)u(t)$ выражаются через производные известной $K(x, \, t)$ и через производные $u(t)$ порядка не более $m$.

    Чтобы оценить их, продифференцируем наше интегральное уравнение:
    \[
        u^{(l)}(x) = f^{(l)}(x) + \int\limits_a^b K^{(l)}_x(x, \, t) u(t)\, \del t.
    \]
    Отсюда можно найти оценку для $u^{(l)}$ через известные $f$ и $K$ и максимум модуля решения. Решение же можно записать, как
    \[
        u = (I - K)^{-1} f \so \|u\| \leqslant \big\|(I - K)^{-1}\big\| \cdot \|f\| \leqslant \dfrac{\|f\|}{1 - \|K\|} \leqslant \dfrac{\|f\|}{1 - \varkappa},
    \]
    где 
    \[
        \varkappa = (b - a) \max \big|K(s, \, t)\big|.
    \]
    Предпоследний переход обусловлен утверждением \ref{st:close-inv}. 

    \begin{rem}
        Во-первых, сейчас у нас все нормы~--- $L^1$, от этого ничего не портится. Во-вторых, мы только что неявно предположили, что $|\varkappa| < 1$.
    \end{rem}

    Получив оценку для модуля решения, мы можем найти оценку
    \[
        \left|\dfrac{\pd^{m}}{\pd t^m} \big(K(x, \, t) u(t)\big)\right| \leqslant M,
    \]
    зависящую только от известных функций.

    Перейдём теперь непосредственно к оценке ошибки. У нас есть два уравнения 
    \begin{align*}
        Au &= f, \quad A = I - K; \\
        \tilde{A} \tilde{u} &= f, \quad \tilde{A} = I - \tilde{K},
    \end{align*}
    где 
    \[
        \tilde{K}\varphi(x) = \sum\limits_{i = 1}^n A_i K(x, \, x_i) \varphi(x_i).
    \]
    Заметим, что
    \[
        \tilde{A}(u - \tilde{u}) = \tilde{A}u - Au \so \|u - \tilde{u}\| \leqslant \boxed{\|\tilde{A}^{-1}\| \, \|\tilde{A}u - Au\|}\,.
    \]
    Оценим норму $\tilde{A}^{-1}$. Для этого сначала оценим норму $\tilde{K}$:
    \[
        \left | \, \sum\limits_{i = 1}^n A_i K(x, \, x_i) \varphi(x_i) \right | \leqslant \max |K| \cdot \|\varphi\| \cdot \sum\limits_{i = 1}^n A_i = (b - a) \max |K| \cdot \|\varphi\|,
    \]
    поэтому $\|\tilde{K}\| \leqslant \varkappa$.

    Отсюда
    \[
        \|\tilde{A}^{-1}\| = \big\|(I - \tilde{K})^{-1}\big\| \leqslant \dfrac{1}{1 - \varkappa}.
    \]
    Теперь оценим $\|\tilde{A}u - Au\|$:
    \[
        \|\tilde{A}u - Au\| = \max \bigg|R\big[K(x, \, t) u(t)\big]\bigg| \leqslant M\delta(n).
    \]
    В конечном итоге находим
    \[
        \boxed{\|u - \tilde{u}\| \leqslant \dfrac{M\delta(n)}{1 - \varkappa}}\,.
    \]

    Подробнее про этот метод можно прочитать в книгах \cite{gavurin} и \cite{comp-krilov-2}.

    \paragraph{Вариационный принцип для ограниченного оператора; метод Ритца для интегрального уравнения II рода}

    \begin{rem}
        В этом параграфе все гильбертовы пространства вещественны.
    \end{rem}

    Основная идея заключается в том, чтобы свести решение уравнения
    \[
        Au = f
    \]
    к минимизации некоторого функционала. 

    \begin{de}
        \ti{Энергетическим} функционалом для такого уравнения называется
        \[
            \tilde{f}(u) = ( Au, \, u ) - 2(f, \, u).
        \]
    \end{de}

    Чтобы работать с энергетическим функционалом, нужны дополнительные ограничения на оператор $A$.

    \begin{de}
        Оператор $A$ называют \ti{положительно определённым}, если $(Au, \, u) \geqslant k^2 (u, \, u)$\footnote{Это необычное название, кажется. Их называют ещё \ti{полуограниченными снизу}.}.
    \end{de}

    \begin{st} \label{st:semi-bound-inv}
        Самосопряжённый положительно определённый оператор $A$ обратим.
        \begin{proof}
            Положим в доказательстве $k^2 = 1$, ибо на обратимость это не влияет, можно просто разделить $A$ на $k^2$. Заметим, что $\ker A = \{0\}$, поскольку
            \[
                Au = 0 \so (Au, \, u) = 0 \so (u, \, u) = 0 \so u = 0.
            \]
            При этом ортогональное дополнение образа $A$~--- его ядро:
            \[
                x \in \im A^{\perp} \eqv \all u \quad 0 = (x, \, Au) = (Ax, \, u) \eqv Ax = 0.
            \]
            Поэтому
            \[
                \ov{-}{\im A} = \ker A^{\perp} = H,
            \]
            и образ оператора $A$ плотен в $H$.

            Докажем, что он на самом деле равен $H$. Для этого нам пригодится неравенство
            \[
                \|u\|^2 \leqslant (Au, \, u) \leqslant \|Au\| \, \|u\| \so \boxed{\|u\| \leqslant \|Au\|}\,.
            \]
            Пусть $y \in H$. Поскольку образ плотен, найдётся последовательность $\{x_n\}$ такая, что $Ax_n \to y$. Однако
            \[
                \|x_n - x_m\| \leqslant \|Ax_n - Ax_m\|,
            \]
            поэтому $\{x_n\}$ сходится в себе; гильбертово пространство полно, поэтому $x_n \to x$. Но оператор $A$ непрерывен, и
            \[ 
                x_n \to x \so A x_n \to Ax \so Ax = y.
            \]
            Таким образом, $A$ сюръективен, и у него есть теоретико-множественный обратный.

            При этом
            \[
                \|A^{-1}y\| \leqslant \|y\|, 
            \]
            поэтому обратный оператор ограничен.


        \end{proof}
    \end{st}

    \begin{st}
        Если $A$~--- самосопряжённый и положительно определённый, то существует единственное решение $u^*$ уравения $Au = f$, которое совпадает с единственным минимумом энергетического функционала.
        \begin{proof}
            Существование и единственность решения следуют из обратимости оператора. Посчитаем значение функционала на векторе $u^* + h$:
            \begin{align*}
                \tilde{f}(u^* + h) &= \big(A(u^* + h), \, u^* + h\big) - 2(f, \, u^* + h) =  \tilde{f}(u^*) + (Au^*, \, h) + (Ah, \, u^*) + (Ah, \, h) - 2(f, \, h) = \\ &= \tilde{f}(u^*) + (h, \, f) - (f, \, h) + (Ah, \, h).
            \end{align*}
            Мы считаем всё вещественным, поэтому $(h, \, f) = (f, \, h)$ и
            \[
                \tilde{f}(u^* + h) = \tilde{f}(u^*) + (Ah, \, h) \geqslant \tilde{f}(u^*).
            \]
        \end{proof}
    \end{st}

    Метод Ритца устроен примерно так:
    \begin{enumerate}
        \item Выбрать в пространстве $H$ линейно независимый набор $\{\varphi_k\}$.
        \item Рассмотреть конечномерное подпространство $H_n$, натянутое на первые $n$ векторов базиса.
        \item Найти в нём минимум функционала $\tilde{f}$ и считать его приближением.
    \end{enumerate}

    Минимум в $H_n$ будем искать в виде
    \[
        u_n = \sum\limits_{k = 1}^n c_k \varphi_k.
    \]

    \begin{st}
        Координаты $c_n$ минимума $\tilde{f}$ в подпространстве $H_n$ находится из системы линейных уравнений
        \[
            \sum\limits_{k = 1}^n (A \varphi_k, \, \varphi_i) \, c_n = (f, \, \varphi_i)
        \]
        \begin{proof}
            Если подставить 
            \[
                u_n = \sum\limits_{k = 1}^n c_k \varphi_k
            \]
            в формулу для функционала
            \[
                \tilde{f}(u_n) = ( Au_n, \, u_n ) - 2(f, \, u_n),
            \]
            получится
            \[
                \tilde{f}(u_n) = \sum\limits_{k, \, m} c_k c_m (A\varphi_k, \, \varphi_m) - 2\sum\limits_m c_m (f, \, \varphi_m).
            \]
            Дифференцируя это выражение по $c_i$ и приравнивая к нулю, получим нужную СЛУ.
        \end{proof}
    \end{st}

    \begin{rem}
        Симметричная матрица $a_{ij} = (A\varphi_i, \, \varphi_j)$~--- матрица Грама положительно определённой симметрической билинейной формы $g(u, \, v) = (Au, \, v)$. Известно, что определитель матрицы Грама равен квадрату объёма параллелипипеда, натянутого на базисные вектора, в соответствующей метрике. Он, конечно, ненулевой, а потому система линейных уравнений разрешима однозначно.
    \end{rem}

    Поговорим о сходимости метода Ритца.

    \begin{st}\label{st:rietz-conv-1}
        Если набор $\{\varphi_k\}$ таков (это по сути означает, что он является базисом), что
        \[
            \all v \in H \quad \all \varepsilon > 0 \quad \ex n, \, \alpha_i \, \col \; \left\|v - \sum\limits_{i = 1}^n \alpha_i \varphi_i \right\| < \varepsilon,
        \]
        то метод Ритца сходится, т.е. $\|u_n - u^*\| \to 0$.
        \begin{proof}
            Поскольку оператор $A$ положительно определён, форма $g(u, \, v) = (Au, \, v)$ является настоящим скалярным произведением. Мы утверждаем, что $u_n$~--- элемент из $H_n$, ближайший к $u^*$ с точки зрения метрики $g$. Докажем это. Для этого предположим, что
            \[
                u_n = v_n + h,
            \]
            где $v_n = u^* - v_n^{\perp}$~--- ближайший к $u^*$ элемент из $H_n$, а $v_n^{\perp} \perp H_n$. Тогда
            \begin{align*}
                \tilde{f}(u_n) &= \tilde{f}(u^*) + \big(A(h - v_n^\perp), \, h - v_n^\perp\big) = \\ &= \tilde{f}(u^*) + g(v_n^{\perp}, \, v_n^{\perp}) + g(h, \, h).
            \end{align*}
            Видно, что это выражение минимально, когда $h = 0$ и $u_n = u^* - v_n^{\perp}$.

            Найдём теперь по $\varepsilon$ такое $N$ и $w \in H_N$, что $\|w - u^*\| < \varepsilon$. Тогда
            \[
                \|u_n - u^*\| \leqslant \dfrac{1}{k} \, \|u_n - u^*\|_A \leqslant \dfrac{1}{k} \|w - u^*\|_A \leqslant \dfrac{\sqrt{\|A\|}}{k} \|w - u^*\| < \dfrac{\sqrt{\|A\|}}{k} \varepsilon,
            \]
            где $\|x\|_A = \sqrt{g(x, \, x)}$. Объясним переходы по пунктам:
            \begin{enumerate}
                \item Потому что $g(x, \, x) \geqslant k^2 (x, \, x)$.
                \item Потому что $u_n$~--- самый близкий элемент к $u^*$.
                \item Потому что $(Ax, \, x) \leqslant \|Ax\| \, \|x\| \leqslant \|A\| \, (x, \, x)$.
                \item Прост)00
            \end{enumerate}
            Эпсилон домножился на константу, но это не страшно: стремление к нулю всё равно есть.
        \end{proof}
    \end{st}

    \begin{rem}
        Видно, что скорость сходимости метода от гладкости ядра не зависит (только от его нормы). По сути она определяется тем, насколько быстро убывают коэффициенты разложения $u^*$ по базису $\varphi_k$, что связано с гладкостью решения. Зато ограничения на оператор сильные.
    \end{rem}

    \paragraph{Интегральное уравнение I рода и его некорректность}

    \begin{de}
        \ti{Интегральным уравнением I рода} называют уравнение вида
        \[
            \int\limits_a^b K(x, \, t) u(t) \, \del t = f(x).
        \]
    \end{de}

    \begin{de}
        Говорят, что задача \ti{корректна}, если при малых изменениях исходных данных решение меняется слабо.
    \end{de}

    \begin{de}
        Задачу вида $Au = f$ называют \ti{корректной}, если у оператора $A$ есть ограниченный обратный.
    \end{de}

    Кажется, эти два определения почти одинаковые. :)

    \begin{st}
        Задача о решении уравнения Фредгольма I рода некорректна.
        \begin{proof}
            Интуитивно это понятно: мы решаем уравнение вида $Ku = f$, где $K$~--- компактный оператор. Его образ маленький, и логично, что слегка изменив $f$ мы можем получить задачу с совсем другим решением или, скорее, вовсе неразрешимую. 

            Покажем это для случая симметричного ядра. Симметричность позволит нам выбрать собственный базис $\{\varphi_n\}$ с собственными числами $\lambda_n$. Пусть
            \[
                u = \sum u_i \varphi_i; \quad u = \sum f_i \varphi_i.
            \]
            Тогда уравнение перепишется, как
            \[
                \sum u_i \lambda_i \varphi_i = \sum f_i \varphi_i \eqv \boxed{u_i \lambda_i = f_i}\,.
            \]
            Формально решение имеет вид
            \[
                u = \sum\limits_{i = 1}^{\infty} \dfrac{f_i}{\lambda_i}\varphi_i.
            \]
            Если $\lambda_i = 0$, а $f_i \neq 0$, то задача наверняка не имеет решения. Всё плохо, даже если это не так: известно, что собственные числа компактного оператора стремятся к нулю, поэтому ряд для $u$ будет сходиться, только если $f_i$ убывают ещё быстрее.

            Посмотрим, что будет при небольшом изменении начальных данных; пусть
            \[
                Ku = f; \quad k\tilde{u} = \tilde f; \quad \tilde{f} = f + \delta f; \quad \tilde{u} = u + \delta u,
            \]
            причём $\|\delta f\| < \varepsilon$. Функция $\delta u$ удовлетворяет уравнению $K\delta u = \delta f$. Решение должно выглядеть как
            \[
                \delta u = \sum\limits_{i = 1}^{\infty} \dfrac{\delta f_i}{\lambda_i}\varphi_i.
            \]
            Даже если этот ряд сходится, нельзя гарантировать, что при $\varepsilon \to 0$ $\delta u$ тоже будет стремиться к нулю. 

            Действительно, всегда можно выбрать $\delta f = \varepsilon \varphi_n$, где $n$ таково, что $\lambda_n < \varepsilon$. Тогда $\|\delta u\|$ будет больше $1$.
        \end{proof}

        Подробнее про это можно прочитать в книге \cite{bahvalov}.
    \end{st}

    \paragraph{Условная корректность по Тихонову, метод квазирешений}

    \begin{thm}(об условной корректности)
        Пусть оператор $A$ на гильбертовом пространстве $H$ некорректен, но инъективен (устанавливает взаимно однозначное отображение на образ). Рассмотрим компакт $L \in H$, пусть $M$~--- его образ. Отображение $A^{-1}$ непрерывно на $M$\footnote{На самом деле это просто стандартная теорема про то, что непрерывное отображение из компактного пространства в хаусдорфово является гомеоморфизмом на образ.}.
        \begin{proof}
            Возьмём какую-нибудь последовательность $f_n \to f$ в $M$. Оператор $A$ инъективен, поэтому элементы $u_n$ такие, что $Au_n = f_n$ определены однозначно. Выберем в $\{u_n\}$ какую-нибудь сходящуюся подпоследовательность $u_n' \to u'$. 

            Поскольку оператор $A$ непрерывен, $Au_n' \to Au'$; но $Au_n' \to f$, поэтому и $Au' = f \so u' = A^{-1}f$. Но тогда выходит, что пределы всех сходящихся подпоследовательностей в $\{u_n\}$ одинаковы! Поэтому все частичные пределы совпадают, и $u_n$ имеет предел, который равен $A^{-1}f$, что и даёт нам непрерывность. 
        \end{proof}
    \end{thm}

    \begin{de}
        Это свойство~--- иметь непрерывный обратный на образах компактов~--- и называется \ti{условной корректностью}.
    \end{de}

    Пусть мы решаем задачу $Au = f$, причём правая часть известна с погрешностью:
    \[
        \|f - f_{\delta}\| \leqslant \delta,
    \]
    но уравнение $Au = f_{\delta}$ не всегда имеет решение даже когда $f_{\delta}$ из этого шара. Приходим к определению \ti{квазирешения}:

    \begin{de}
        Зафиксируем конкретную $f_{\delta}$. Тогда \ti{квазирешением} уравнения $Au = f_{\delta}$ называется вектор $u_{\delta}$, при котором достигается 
        \[
            \min\limits_{u \in D} \|Au - f_{\delta}\|, \quad D = \big\{u \, \big| \, \|u\| \leqslant R\big\}.
        \]
    \end{de}
    
    %unsure
    Если искать не при $\|u\| \leqslant R$, а при $\|u\| = R$, получится задача на условный экстремум.
    %unsure

    Используя метод множителей Лагранжа, будем минимизировать функционал
    \[
        F(u) = \alpha \|u\|^2 + \|Au - f_{\delta}\|^2.
    \]
    \begin{st}
        Минимум этого функционала удовлетворяет уравнению
        \[
            (\alpha I + A^* A)u = A^* f.
        \]
        \begin{proof}
            Обычный поиск вариации, нужно расписать $F(u + th)$ через скалярные произведения, продифференцировать по $t$, а после положить $t$ равным нулю.
        \end{proof}
    \end{st}

    Мы получили уравнение, похожее на исходное, но оно уже второго типа, а при малых $\alpha$ похоже на исходное. Произошла \ti{регуляризация}! Более того, оператор $\alpha I + A^*A$ самосопряжён, и

    \[
        \big((\alpha I + A^*A)u, \, u\big) = \alpha(u, \, u) + (Au, \, Au) \geqslant \alpha(u, \, u),
    \]
    поэтому применим вариационный принцип.

    % \begin{rem}
    %     Насколько я понимаю, метод квазирешений нам рассказан в основном для того, чтобы прийти к регуляризации. Из квазирешений следует, что при некотором $\alpha$ минимум функционала должен хорошо приближать решение~--- мотивация! Плюс демонстрация того, что не совсем очевидно, что альфу можно просто к нулю стремить.
    % \end{rem}


\paragraph{Метод регуляризации для уравнения I рода, сходимость}

    % \begin{rem}
    %     Кажется, этого билета почти нет у Оли, поэтому я опускаю доказательства. Это хорошо написано у Ангелины.
    % \end{rem}

    \begin{rem}
        В этом билете доказательств нет, их надо бы дописать. Они хорошо восстанавливаются по бумажным конспектам.
    \end{rem}

    Идея регуляризации заключается в том, чтобы минимизировать функционал вида
    \[
        F(u) = \alpha \Omega(u) + \|Au - f\|^2,
    \]
    где $\Omega(u) \geqslant 0$ и множества $\Omega(u) < C$ компактны.

    \begin{rem}
        В методе квазирешений у нас получился $\Omega(u) = \|u\|^2$, для него эти множества~--- открытые шары, они совсем не компактны.
    \end{rem}

    Стандартный выбор~--- функционал

    \[
        \Omega(u) = \int\limits_a^b u'^2 \, \del t.
    \]

    Правда, при этом мы начинаем искать решение среди гладких функций.

    \begin{st}
        Для такого функционала $\Omega$ множества $\Omega(u) < C$ компактны.
        \begin{proof}
            %TODO
            Стандартное рассуждение, использующее теорему Арцела-Асколи: подмножество в пространстве непрерывных функций на отрезке компактно тогда и только тогда, когда оно равномерно ограничено и равностепенно непрернывно. Из компактности в смысле топологии пространства непрерывных функций следует компактность в смысле $L^2$-нормы.
        \end{proof}
    \end{st}

    Годятся и функционалы

    \[
        \Omega(u) = \int\limits_a^b u^{(p)^2} \, \del t, \quad \Omega(u) = \int\limits_a^b u'^2 - u^2 \, \del t.
    \]

    \begin{thm}
        Пусть $\|f - f_{\delta}\| < \delta$, и мы решаем приближённую задачу $A\tilde{u} = f_{\delta}$ вместо точной. Если $\delta$ и $\alpha$ стремятся к нулю так, что 
        \[
            \dfrac{\delta^2}{\alpha} \leqslant \gamma < \infty,
        \]
        то $\tilde{u} \to u$.
        \begin{proof}
            См. бумажный конспект.
        \end{proof}
    \end{thm}

    Коэффициент $\alpha$ обычно подбирают эмпирически: если он мал, то решение будет ближе к $\tilde{u}$, если велик, оно будет глаже... Стандартный функционал приводит к вариационной задаче
    \[
        K^*Ku - \alpha u'' = K^* f.
    \]

\end{document}
% vim:wrapmargin=3


\chapter{Вариационные методы}
\label{chap:varm}
\documentclass{trlnotes}
\setlayout{hardcopy}
\usepackage{silence}
\WarningFilter{latex}{Reference}
\graphicspath{{../../img/}}

\begin{document}
    \paragraph{Вариационный принцип для уравнения с неограниченным оператором}

    \begin{de}
        \ti{Неограниченным} называется оператор $A$ на гильбертовом пространстве, определённый на его всюду плотном линейном подпространстве $\mc{D}(A)$.
    \end{de}

    Мы будем требовать от $A$ также симметричности и положительной определённости.

    \begin{de}
        Билинейная форма $(u, \, v)_A = (Au, \, v)$ называется \ti{энергетическим скалярным произведением}, норма $\|u\|_A = (u, \, u)_A$~--- \ti{энергетической нормой}. Пополнение $H_A$ пространства $\mc{D}(A)$ по энергетической норме называется \ti{энергетическим пространством}. 
    \end{de}

    \begin{rem}
        В доказательстве \ref{st:rietz-conv-1} мы видели, что
        \[
            \|u\| \leqslant k^{-1} \|u\|_A.
        \]
        Поэтому если последовательность сходится в себе по энергетической норме, то она сходится и по обычной; кофинальные последовательности тоже одинаковые и там, и там. Поэтому пополнение по энергетической норме можно рассматривать, как подмножество $H$.
    \end{rem}

    \begin{thm}(О вариационном принципе)
        Рассмотрим энергетический функционал $F(u) = (u, \, u)_A - 2(f, \, u)$.
        \begin{enumerate}
            \item $F(u)$ имеет единственный минимум $u^*$.
            \item Если $u^* \in \mc{D(A)}$, то $Au^* = f$.
            \item Если $Au_0 = f$, то $u^* = u_0$.
        \end{enumerate}
        \begin{proof}
            Рассмотрим функционал $\Phi(u) = (f, \, u)$. Он ограничен на $H_A$, поскольку
            \[
                \big|(f, \, u)\big| \leqslant \|f\| \, \|u\| \leqslant k^{-1} \|f\| \, \|y\|_A.
            \]
            По теореме Рисса он представим в виде $\Phi(u) = (u^*, \, u)_A$.
            Тогда
            \[
                F(u) = (u, \, u)_A - 2(u^*, \, u)_A = \|u - u^*\|^2 - \|u^*\|^2.
            \]
            Ясно, что минимум достигается при $u = u^*$.

            Третий пункт очевиден, ибо
            \[
                (f, \, u) = (Au_0, \, u) = (u_0, \, u)_A \so u_0 = u^*
            \]
            То же самое в обратную сторону даёт пункт 2.
        \end{proof}
    \end{thm}

    \paragraph{Метод Ритца, сходимость}

    Метод Ритца для неограниченных операторов похож на обычный. $\varphi_n$ теперь лежат в $H_A$, а в остальном~--- всё то же самое.

    \begin{thm}
        Если набор $\{\varphi_k\}$ таков (это по сути означает, что он является базисом), что
        \[
            \all v \in H \quad \all \varepsilon > 0 \quad \ex n, \, \alpha_i \, \col \; \left\|v - \sum\limits_{i = 1}^n \alpha_i \varphi_i \right\|_A < \varepsilon,
        \]
        то метод Ритца сходится, т.е. $\|u_n - u^*\|_A \to 0$.
        \begin{proof}
            Доказательство теоремы аналогично доказательству \ref{st:rietz-conv-1}. Первый абзац такой же, по сути, а дальше надо оставить только оценки, содержащие энергетическую норму.
        \end{proof}
    \end{thm}

    \paragraph{Метод Ритца для обычной краевой задачи, вид энергетического пространства, естественные граничные условия}

    Рассмотрим краевую задачу для уравнения 
    \[
        L(y)(x) = -(p(x)y')' + q(x)y = f(x)
    \]
    на отрезке $[a, \, b]$ с граничными условиями
    \begin{itemize}
        \item I типа: $y(a) = 0$, $y(b) = 0$.
        \item III типа: $y'(a) = \alpha y(a)$, $y'(b) = \beta y(b)$.
    \end{itemize}

    \begin{de}
        \ti{Классическое решение}~--- лежит в $C^2\big([a, \, b]\big)$, удовлетворяет уравнению в каждой точке. 
    \end{de}

    Ну и $\mc{D}(L) = C^2\big([a, \, b]\big)$. 

    \begin{rem}
        На самом деле, мы ищем решения не в $\mc{D}(L)$, а в более узких пространствах. В случае условия I типа нас интересует простанство
        \[
            D_{I} = \big\{y \in \mc{D}(L) \, \big| \, y(a) = y(b) = 0\big\},
        \]
        а в случае условия III типа
        \[
            D_{III} = \big\{y \in \mc{D}(L) \, \big| \, y'(a)\alpha y(a), \; y'(b) = \alpha y(b)\big\}.
        \]
        Именно их мы будем пополнять, создавая соответствующее энергетическое пространство.
    \end{rem}

    \begin{st}
        Если $\alpha \geqslant 0$ и $\beta \leqslant 0$ в добавок к условиям
        \[
            p(x) \geqslant p_0 > 0, \quad q(x) \geqslant 0,
        \]
        то оператор получится симметричный и положительно определённый.

        \begin{proof}
            Посмотрим, как будет выглядеть энергетическое скалярное произведение:

            \[
                (Ly, \, z) = \int\limits_a^b \big(-(py')' + qy\big) z \, \del x = -py'z\big|_a^b + \int\limits_a^b (py'z' + qyz) \, \del z.
            \]
            Интеграл обозначим через $I(X)$, а внеинтегральный член~--- $Q(x)$. Если условия первого типа, то $Q = 0$, а если третьего, то
            \[
                Q(y, \, z) = -\beta p(b)y(b)z(b) + \alpha p(a)y(a)z(a).
            \]
            Симметричность уже видна, и
            \[
                (Ly, \, y) = \int\limits_a^b (py'^2 + qy^2) \, \del x -\beta p(b)y(b)^2 + \alpha p(a)y(a)^2.
            \]
            Если, $\alpha \geqslant 0$ и $\beta \leqslant 0$, то и положительная определённость будет.
        \end{proof}
    \end{st}

    \begin{de}
        \ti{Пространством Соболева} $W_p^k(Q) \subset L^p(Q)$ называют пространство функций, обобщённые производные которых вплоть до $k$-й лежат в $L_p(Q)$.
    \end{de}

    \begin{rem}
        На пространствах Соболева есть норма. Нас будет интересовать пространство $W_2^1\big([a, \, b]\big)$; на нём эта норма имеет вид
        \[
            \|f\|^2_{W_2^1} = \int\limits_a^b (f^2 + f'^2) \, \del x.
        \]
        Можно доказать, что с такой нормой является гильбертовым (а произвольные пространства Соболева~--- банаховы). 
    \end{rem}

    \begin{st}
        Энергетическая норма для оператора $L$ эквивалентна норме в $W_2^1$.
        \begin{proof}
            Пусть $P_m = \max p$, $Q_m = \max q$, $M = \max(P_m, \, Q_m)$. Нетрудно доказывается, что $\|y\|_{W_2^1} \leqslant C \|y\|_L$:
            \[
                \int\limits_a^b (f^2 + f'^2) \, \del x \leqslant \dfrac{1}{M} \int\limits_a^b (pf^2 + qf'^2) \, \del x \leqslant \dfrac{1}{M} \|f\|_L.
            \]
            Обратное утверждение очевидно для I типа граничных условий:
            \[
                \int\limits_a^b (pf^2 + qf'^2) \, \del x \leqslant M \|f\|_{W_2^1}.
            \]
            Чтобы разобраться с граничными условиями III типа, нам понадобится лемма:
            \begin{lm}
                Для любой точки $x$ значение $y(x)^2$ не превосходит константы, умноженной на $\|y\|_{W_2^1}^2$.
                \begin{proof}
                    Ограничение соболевской нормы даёт ограничение на интеграл от квадрата функции + не позволяет ей расти слишком быстро, поэтому есть надежда, что значения и правда будут ограничены нормой. Займёмся оценкой. Очевидно, что
                    \[
                        y(x) = y(\xi) + \int\limits_{\xi}^x y(t) \, \del t.
                    \]
                    Поскольку $(a + b)^2 \leqslant 2(a^2 + b^2)$,
                    \[
                        y(x)^2 \leqslant 2y(\xi)^2 + 2 \left(\,\int\limits_{\xi}^x y'(t) \, \del t\right)^2.
                    \]
                    При этом интеграл
                    \[
                        \int\limits_{\xi}^x y'(t) \, \del t
                    \]
                    является $L^2$-произведением (в отрезке от $\xi$ до $x$) $\big(y'(t), \, 1\big)_{L^2}$, и
                    \[
                        \big(y'(t), \, 1\big)_{L^2}^2 \leqslant \|1\|_{L^2([x, \, \xi])}^2 \,\|y'\|_{L^2([x, \, \xi])}^2 = (x - \xi) \int\limits_{\xi}^x y'(t)^2 \, \del t \leqslant (b - a)\|y'\|^2_{L^2}
                    \]
                    В итоге получаем, что
                    \[ 
                        y(x)^2 \leqslant 2y(\xi)^2 + 2(b - a) \|y'\|^2_{L^2}
                    \]
                    Навесив слева и справа интегралы по $\xi$, получим, что
                    \[
                        y(x)^2 \leqslant \dfrac{2}{b - a} \|y\|^2_{L^2} + 2(b - a) \|y'\|^2_{L^2} \leqslant C \|y\|^2_{W_2^1}.
                    \]
                \end{proof}
            \end{lm}
            Используя полученную оценку, нетрудно оценить отвечающий за граничные условия член $Q(x)$ через соболевскую норму.
        \end{proof}
    \end{st}

    \begin{rem}
        Ещё выполняется \ti{теорема вложения}: все функции из $W_2^1$ непрерывны, при этом отображение вложения $W_2^1 \to C\big([a, \, b]\big)$ непрерывно.
    \end{rem}

    \begin{st}
        Энергетическое пространство $H_{L}$ является подпространством в $W_2^1$.
        \begin{proof}
            Не очень важно, $D_{I}$ или $D_{III}$ придётся пополнять: они оба лежат в $\mc{D}(L)$, про которое мы доказали, что с энергетической нормой оно гомеоморфно вкладывается в $W_2^1$. Поскольку $W_2^1$ гильбертово, пополнение нас из него не выведет.
        \end{proof}
    \end{st}

    \begin{rem}
        Пополнение пространства $D_I$ приведёт нас к пространству $\overset{\circ}{W_1^2}$ элементов $W_1^2$, удовлетворяющих граничному условию I типа. С условием III так не получится, поскольку
        %unsure
        производная~--- не непрерывный функционал, и мы придём ко всему $W_1^2$.
        %unsure
        По этой причине условия I типа называют \ti{главными}, а III типа~--- \ti{естественными}.
    \end{rem}

    О пространствах Соболева в контексте вычислительных методов можно прочитать в книге \cite{lebedev}, и ещё подробнее в книге \cite{atkinson}.

    \paragraph{ВРМ-1 для обычной краевой задачи}

    Идея \ti{вариационно-разностных методов} заключается в том, чтобы использовать сетку и минимизацию функционала одновременно. 

    Пусть в сетке $n$ элементов, $h = \dfrac{b - a}{n}$, $x_k = a + k h$; рассмотрим пространство, состоящее из сеточных функций $y_{(n)} = \{y_k\}_0^n$. Суть ВРМ-1 в том, чтобы заменить интегралы на суммы, а производные~--- на разности, и минимизировать функционал на сеточных функциях.

    Наш функционал имеет вид
    \[
        F(y) = (y, \, y)_L - 2(f, \, y) = \int\limits_a^b (py'^2 + qy^2 - 2fy) \, \del x - \beta p(b)y^2(b) + \alpha p(a)y^2(a). 
    \]

    Сделаем численные замены:
    \begin{align*}
        &\int\limits_a^b py' \, \del x \approx h \sum\limits_{k = 0}^{n - 1} p\left(x_k + \dfrac{h}{2} \right) \cdot \left(\dfrac{y_{k + 1} - y_k}{h}\right)^2; \\ 
        &\int\limits_a^b (qy^2 - 2fy) \, \del x \approx k\sum\limits_{k = 0}^{n - 1}\,\!^{'} (q_ky_k^2 - 2f_ky_k),
    \end{align*}
    где сумма со штрихом означает, что это формула трапеций (т.е. крайние слагаемые домножены на $\nicefrac{1}{2}$).

    Не представляет труда теперь выписать сеточный функционал. Далее минимум ищется дифференцированием по $y_k$ и приравниванием всех производных к нулю. В итоге для внутренних точек получаются уравнения
    \[
        -\dfrac{1}{h}\left(p_{i + \tfrac{1}{2}} \dfrac{y_{i + 1} - y_i}{h} - p_{i - \tfrac{1}{2}} \dfrac{y_i - y_{i - 1}}{h}\right) + q_i y_i = f_i.
    \]
    Они напоминают уравнения разностной прогонки.
    %unsure на них бы сослаться

    Для левого конца получится уравнение
    \[
        -p_{\frac{1}{2}} \dfrac{y_1 - y_0}{2} + \dfrac{h}{2}(q_0 y_0 - f_0) + \alpha p_0 y_0 = 0
    \]
    Второе слагаемое неожиданное! Ведь здесь стоило ожидать простейшее приближение $y'(a) = \alpha y(a)$. Оказывается, что оно компенсирует сдвиг:
    \begin{align*}
        p_{\frac{1}{2}} \dfrac{y_1 - y_0}{2} &= [py']\left(a + \dfrac{h}{2}\right) + O(h^2)  = \\ &= p(a)y'(a) + \dfrac{h}{2}(py')'|_a + O(h^2) = \\ &= p(a)y'(a) + \dfrac{h}{2} \big(q(a)y(a) - f(a)\big) + O(h^2). 
    \end{align*}

    \paragraph{ВРМ-2 для обычной краевой задачи}

    Идея ВРМ-2 заключается в том, чтобы <<поднять>> сеточные функции до каких-нибудь функций из $W_2^1$  (с помощью некоторого сорта интерполяции), а потом минимизировать функционал на получившемся пространстве. 

    Будем работать с граничными условиями I типа.

    Используем кусочно-линейную интерполяцию:
    \[
        \tilde{y}_{(n)}(x) = \dfrac{x_{k + 1} - x}{h} y_k + \dfrac{x - x_k}{h}y_{k + 1}.
    \]
    Производная определена всюду, кроме узлов:
    \[
        \tilde{y}_{(n)}'(x) = \dfrac{y_{k + 1} - y_k}{h}.
    \]

    %unsure
    Однако узлы~--- множество меры ноль, поэтому производная всё равно определена, как обобщённая функция. Можно доказать, что это и будет производная в смысле обобщённых функций от восполненной сеточной функции. Поэтому наши восполненные функции находятся в $W_2^1$.
    %unsure

    Можно ввести базисные функции~--- это восполнения сеточных функций, которые равны нулю всюду, кроме одной точки, а в ней равны единица, т.е.
    \[
        \psi_k(x) = \begin{cases}
            \dfrac{x_{k + 1} - x}{h}, \; [x_k, \, x_{k + 1}]; \\ 
            \dfrac{x - x_{k - 1}}{h}, \; [x_{k - 1}, \, x_k]; \\
            0.
        \end{cases}
    \]

    Получилось что-то очень похожее на метод Ритца, но только теперь у нас не фиксированный бесконечный набор $\{\varphi_k\}$, а для каждого $n$ есть набор $\{\psi_k\}$ с понятным геометрическим смыслом.

    Уравнение для минимизации получится такое же:
    \[
        \sum\limits_{k = 1}^{n - 1}(\psi_k, \, \psi_m)_A y_k = (f, \, \psi_m),
    \]
    матрица системы~--- $\{a_{km}\} = (\psi_k, \, \psi_m)_A$.

    Носители базисных функций почти не пересекаются, поэтому
    \[
        |k - m| > 1 \so a_{km} = 0.
    \]
    Поэтому система уравнений снова выходит трёхдиагональной:
    \[
        a_m y_{m - 1} + b_m y_m + a_{m + 1} y_{m + 1} = (f, \, \psi_m),
    \]
    где
    \[
        a_m = a_{m-1, \, m} \text{ и } b_m = a_{mm}.
    \]

    %unsure
    На негладких решениях мы не получим точности лучше, чем $O(h)$. Однако этот метод для них надёжнее, чем просто сеточный.
    %unsure

    \paragraph{Метод Ритца для эллиптического уравнения, энергетическое пространство и естественные условия}

    Рассмотрим уравнение
    \[
        Lu = - \sum\limits_{i = 1}^n \dfrac{\pd}{\pd x_i}\left(a_{ik} \dfrac{\pd u}{\pd x_k}\right) + au = f.
    \]

    Коэффициенты должны удовлетворять нескольким условиям:

    \begin{enumerate}
        \item Все функции действуют в области $\Omega \subset \R^n$. В классическом сеттинге решение считается дважды непрерывно дифференцируемыми в $\Omega$ и непрерывным на $\ov{-}{\Omega}$; $a_{ij}$~--- один раз непрерывно дифференцируемы на $\ov{-}{\Omega}$, а остальные коэффициенты просто непрерывны.
        \item Симметричность (эллиптичность): $a_{ik}(x) = a_{ki}(x)$.
        \item Положительная определённость:
        \[
            \sum\limits_i \sum\limits_k a_{ik} \xi_i \xi_k \geqslant k^2 \sum\limits_i \xi_i^2.
        \]
    \end{enumerate}

    Граничные условия бывают
    \begin{enumerate}
        \item $u|_{\pd\Omega} = 0$~--- I типа (задача Дирихле).
        \item Пусть
        \[
            \dfrac{\pd u}{\pd \nu} \bigg|_{\pd \Omega} = \sum\limits_{i = 1}^n a_{ij} \cos(n, \, x_i) \dfrac{\pd u}{\pd x_i} \bigg|_{\pd \Omega}.
        \]
        Этот оператор называется \ti{конормальной производной}.
        Тогда граничное условие выглядит, как
        \[
            \dfrac{\pd u}{\pd \nu} = \sigma u|_{\pd \Omega}.
        \]
        Это~--- задача третьего рода (задача Фон-Неймана).
        \item Задача второго рода~--- когда $\sigma = 0$.
    \end{enumerate}

    Найдём вид энергетического произведения.

    \begin{st}
        \[
            (Lu, \, v) = \int\limits_{\Omega} \left(\sum\limits_{i, \, j = 1}^n a_{ij} \dfrac{\pd u}{\pd x_j} \dfrac{\pd v}{\pd x_i} + auv\right) \pd x + \int\limits_{\pd \Omega} \sigma u v \, \del S.
        \]
        \begin{proof}
            \[
                (Lu, \, v) = \int\limits_{\Omega}  \left( - \sum\limits_{i = 1}^n \dfrac{\pd}{\pd x_i}\left(a_{ik} \dfrac{\pd u}{\pd x_k}\right) + au \right) v \, \del x
            \]
            Используя формулу интегрирования по частям
            \[
                \int\limits_{\Omega} \dfrac{\pd u}{\pd x_i} \varphi \, \del x = - \int\limits_{\Omega} u  \dfrac{\pd \varphi}{\pd x_i} \, \del x + \int\limits_{\pd \Omega} u\varphi \cos(n, \, x_i) \, \del S,
            \]
            найдём искомый результат.
        \end{proof}
    \end{st}

    \begin{st}
        $\hphantom{.}$
        Оператор положительно определён, если
        \begin{enumerate}
            \item Задача первого типа: $a(x) \geqslant 0$;
            \item Задача второго типа: $a(x) \geqslant a_0 > 0$;
            \item Задача третьего типа: $a(x) \geqslant a_0 > 0$, $\sigma(x) \geqslant 0$ или $a(x) \geqslant 0$, $\sigma(x) \geqslant \sigma_0 > 0$.
        \end{enumerate}
        \begin{proof}
            \[
                (Lu, \, u) = \int\limits_{\Omega} \left(\sum\limits_{i, \, j = 1}^n a_{ij} \dfrac{\pd u}{\pd x_j} \dfrac{\pd u}{\pd x_i} + au^2\right) \pd x + \underbrace{\int\limits_{\pd \Omega} \sigma u^2 \, \del S}_{\text{III}}.
            \]
            Нам понадобится неравенство Фридрихса 
            \[
                \int\limits_{\Omega} u^2 \, \del x \leqslant c_1 \left(\;\int\limits_{\Omega} \sum \left(\dfrac{\pd u}{\pd x_i}\right)^2 \, \del x + \int\limits_{\pd \Omega} u^2 \, \del S \right).
            \]
            \begin{enumerate}
                \item Здесь работает совсем грубая оценка:
                \[
                    \int\limits_{\Omega} \left(\sum\limits_{i, \, j = 1}^n a_{ij} \dfrac{\pd u}{\pd x_j} \dfrac{\pd u}{\pd x_i} + au^2\right) \pd x \geqslant \int\limits_{\Omega} \sum\limits_{i, \, j = 1}^n a_{ij} \dfrac{\pd u}{\pd x_j} \dfrac{\pd u}{\pd x_i} \pd x \geqslant k^2 \int\limits_{\Omega} \sum\limits_{i = 1}^n \left(\dfrac{\pd u}{\pd x_j}\right)^2\pd x \geqslant \dfrac{k^2}{c_1} \int\limits_{\Omega} u^2 \, \del x.
                \]
                \item 
                Ещё проще, как это ни странно.
                \[
                    \int\limits_{\Omega} \left(\sum\limits_{i, \, j = 1}^n a_{ij} \dfrac{\pd u}{\pd x_j} \dfrac{\pd u}{\pd x_i} + au^2\right) \pd x \geqslant a_0\int\limits_{\Omega} u^2 \, \del x
                \]
                \item Первый вариант доказывается точно так же, как для II типа, а второй:
                \[
                    \int\limits_{\Omega} \left(\sum\limits_{i, \, j = 1}^n a_{ij} \dfrac{\pd u}{\pd x_j} \dfrac{\pd u}{\pd x_i} + au^2\right) \pd x + \int\limits_{\pd \Omega} \sigma u^2 \, \del S \geqslant k^2 \int\limits_{\Omega} \sum\limits_{i = 1}^n \left(\dfrac{\pd u}{\pd x_j}\right)^2\pd x + \sigma_0 \int\limits_{\pd \Omega} u^2 \, \del S \geqslant \dfrac{\min(k^2, \, \sigma_0)}{c_1} \int\limits_{\Omega} u^2 \, \del x.
                \]
            \end{enumerate}
        \end{proof}
    \end{st}

    \begin{rem}
        Можно доказать, что эта энергетическая норма эквивалентна норме в $W_2^1(\Omega)$:
        \[
            \|v\|_{W_2^1}^2 = \int\limits_{\Omega} \left(\sum \left(\dfrac{\pd u}{\pd x_i}\right)^2 + u^2 \right) \, \del x.
        \]
        Энергетическое пространство для II и III типов совпадёт с $W_2^1$, а для типа I унаследует граничное условие и будет состоять из элементов $W_2^1$, обращающихся в ноль на границе.
    \end{rem}

    \begin{rem}
        Вообще всё это очень похоже на обычную краевую задачу, только многомерную. При подборе базиса $\{\varphi_k\}$ для метода Ритца в задаче I типа нужно как-то заставить $\varphi_k$ обращаться в ноль на границе области $\Omega$, которая может быть некрасивой. Чтобы это сделать, можно найти функцию $\omega(x, \, y)$~--- это на плоскости~--- которая положительна в $\Omega$ и равна нулю на границе. Читатель сможет придумать такие функции для квадрата/круга/сектора круга, но вообще это, видимо, искусство.
    \end{rem}

    По поводу этого билета стоит заглянуть в книгу \cite{lebedev}.

\end{document}
% vim:wrapmargin=3


\let \xto=\oldxypicxto
\chapter{Уравения в частных производных}
\label{chap:pde}
% \documentclass{trlnotes}
% \setlayout{hardcopy}
% \usepackage{silence}
% \WarningFilter{latex}{Reference}
% \graphicspath{{../../img/}}
\documentclass{trlnotes}
\usepackage{trmath}
\addcompatiblelayout{commonplace}
\setlayout{commonplace}
\usepackage{trthm}
\usepackage{trsym} 
\usepackage{trphys}
\input{mdefs}
\usepackage{silence}
\usepackage{tikz}
\WarningFilter{latex}{Reference}
\graphicspath{{../../img/}}
\makeatletter
\let\xymatrix\@gobble
\makeatother
\begin{document}
\paragraph{Разностный метод для общего уравнения теплопроводности, явная схема}

\begin{de}
	Общее \ti{уравнение теплопроводности} выглядит вот так:
	\[
		\dfrac{\pd u}{\pd t} = a_0 \dfrac{\pd^2 u}{\pd x^2} + a_1 \dfrac{\pd y}{\pd x} + a_2 u + f.
	\]
	Функции $a_i$ и $f$ зависят от $x$ и $t$.
\end{de}

Работать будем, как всегда, на отрезке $[a, \, b]$; временной отрезок будет $[0, \, T]$.

\begin{de}
	У уравнения теплопроводности бывает \ti{начальное условие}:
	\[
		u(x, \, 0) = \varphi(x),            
	\]
	а также три типа \ti{граничных условий}
	\begin{enumerate}
		\item $u(a, \, t) = \psi_0(t)$, $u(b, \, t) = \psi_1(t)$.
		\item $\dfrac{\pd u}{\pd x}(a, \, t) = \psi_0(t)$, $\dfrac{\pd u}{\pd x}(b, \, t) = \psi_1(t)$.
		\item $\dfrac{\pd u}{\pd x} - \alpha u \big|_{x = a} = \psi_0(t)$, $\dfrac{\pd u}{\pd x} - \beta u \big|_{x = b} = \psi_1(t)$.
	\end{enumerate}
\end{de}
Сетка характеризуется такими же, как обычно, величинами:
\[
	\begin{array}{lll}
		x_i = a + ih, & h = \dfrac{b - a}{h}, & i \in 0\ldots n; \\
		t_k = k\tau, & \tau = \dfrac{T}{M}, & k \in 0 \ldots M.
	\end{array}
\]

Положим $u_i^k = u(x_i, \, t_k)$ и
\[
	Lu = a_0 \dfrac{\pd^2 u}{\pd x^2} + a_1 \dfrac{\pd y}{\pd x} + a_2 u.
\]
Тогда
\[
	(\tilde{L}u)_i^k = a_0 \dfrac{u_{i+1}^k - 2u_i^k + u_{i - 1}^k}{h^2} + a_1 \dfrac{u_{i + 1}^k - u_{i - 1}^k}{2h} + a_2 u_i^k.
\]
Есть два варианта для производной по времени:
\begin{align}\label{eq:therm-A-B}
				&\text{A:} \quad \dfrac{\pd u}{\pd t}(x_i, \, t_k) \approx \dfrac{u^{k + 1}_i - u^k_i}{\tau}, \\
				&\text{B:} \quad \dfrac{\pd u}{\pd t}(x_i, \, t_k) \approx \dfrac{u^{k}_i - u^{k-1}_i}{\tau}.
\end{align}
Для варианта A получается 
\[
	\boxed{\dfrac{u_i^{k + 1} - u_i^k}{\tau} = \tilde{L}u_i^k + f(x_i, \, t_k)}\,.
\]
Это простейшая явная схема. 

\begin{figure}[h] \label{fig:therm-simple}
	\begin{center}
		\includegraphics[scale=0.9]{../img/pde/therm-simple.pdf}
	\end{center}
	\caption{Простейшая явная схема для уравнения теплопроводности.}
\end{figure}

В таком виде уравнения можно писать для $i \in 1\ldots n-1$, $k\in 0\ldots M-1$; нужны дополнительные с граничными условиями. 

\begin{itemize}
	\item Начальные условия: $u_i^0 = \varphi(x_i)$.
	\item Граничные условия:
		\begin{enumerate}
			\item $u_0^k = \alpha_1(t_k)$, $u_n^k = \alpha_2(t_k)$; при этом выполняются условия согласования \ti{нулевого порядка}
				\[
					\varphi(a) = \alpha_1(0), \quad \varphi(b) = \alpha_2(0).
				\]
			\item Для типов II, III используются такие же трюки, как в обычных диффурах. Надо аппроксимировать производные. Можно применять метод фиктивных точек или метод исключения главного члена погрешности.

				В угловых точках снова возникнет два разных условия:
				\[
					u_0^0 = \varphi(a) \text{ и } \dfrac{\pd u}{\pd x}(a, \, 0) = \beta_1(0) u_0^0 + \alpha_1(0).
				\]
				Будет ли выполняться равенство
				\[
					\varphi'(a) = \beta_1(0) u_0^0 + \alpha_1(0)?
				\]
				Оно называется \ti{условием согласования I порядка}. Без него уравнения не станут формально противоречивы.
		\end{enumerate}
\end{itemize}

Если разрешить уравнения относительно $u_i^{k + 1}$, получится 
\[
	u_{i}^{k + 1} = A_i^k u_{i-1}^k + B_i^k u_i^k + C_i^k u_{i+1}^k + D_i^k.
\]
Коэффициенты выражаются по формулам
\[
	\begin{array}{ll}
		A_i^k = \sigma a_0 - \sigma a_1 \dfrac{h}{2}, & C_i^k = \sigma a_0 + \sigma \dfrac{h}{2} a_1, \\
		B_i^k = 1 - 2 \sigma a_0 + \tau a_2,  &D_i^k = \tau f(x_i, \, t_k),
	\end{array}
\]
где $\sigma = \dfrac{\tau}{h^2}$.

Можно просто двигаться вперёд по \ti{слоям}~--- множествам точек с постоянным временем; значения находятся последовательно.

\paragraph{Неявная схема для уравнения теплопроводности}

Неявная схема получается, если в \ref{eq:therm-A-B} выбрать вариант B. Сверху вниз (т.е. назад по времени) просчитать не получится, поскольку начальные данные даются в начале, а не в конце.

\begin{figure}[h] \label{fig:therm-implicit}
	\begin{center}
		\includegraphics[scale=0.9]{../img/pde/therm-implicit.pdf}
	\end{center}
	\caption{Неявная схема для уравнения теплопроводности.}
\end{figure}

Формулы получатся такие:
\[
	A_i^k u_{i-1}^k - B_i^k u_i^k + C_i^k u_{i+1}^k = D_i^k,
\]
где
\[
	\begin{array}{ll}
		A_i^k = \sigma a_0 - \sigma a_1 \dfrac{h}{2}, & C_i^k = \sigma a_0 + \sigma \dfrac{h}{2} a_1, \\
		B_i^k = 1 + 2 \sigma a_0 - \tau a_2,  &D_i^k = -u_i^{k-1} - \tau f(x_i, \, t_k),
	\end{array}
\]
и $\sigma = \dfrac{\tau}{h^2}$.

По сути, движение всё ещё послойное. Но на каждом слое я не могу просто посчитать значение, используя три значения с предыдущего слоя: наоборот, получается уравнение, которое связывает три значения с текущего слоя с одним уже известным. В итоге получается система с трёхдиагональной матрицей, которая замыкается добавлением граничных условий:
\[
	u_0^k = \alpha_1(t_k), \; u_n^k = \alpha_2(t_k),
\]
если они заданы по первому типу, в противном случае применяются стандартные аппроксимации. 

Система решается методом разностной прогонки \ref{par:ode::fintdma}.

Метод прогонки срабатывает, поскольку
\[
	A_i^k + C_i^k = 2\sigma a_0 = B_i^k + \tau a_2 - 1,
\]
и можно сослаться на \ref{prop:ode::diffeqest::suff}. Наверное, на практике это правда так, потому что $\tau a_2 \ll 1$.



\paragraph{Явная схема для простейшего уравнения теплопроводности, решение разностных уравнений, неустойчивость}

Рассмотрим уравнение
\[
	\dfrac{\pd u}{\pd t} = \dfrac{\pd^2 u}{\pd t^2}, \quad t \in [0, \, \pi].
\]
с начальным условием $u(x, \, 0) = \varphi(x)$ и граничными условиями
\[
	u(0, \, t) = u(\pi, \, t) = 0
\]

Для него разностные уравнения исключительно просты:
\[
	\dfrac{u^{k + 1}_l - u^k_l}{\tau} = \dfrac{u^k_{l + 1} - 2u^k_l + u^k_{l-1}}{h^2}, \quad u_0^k = u_n^k = 0.
\]
При этом
\[
	h = \dfrac{\pi}{n}, \quad x_l = lh, \quad u_l^0 = \varphi(x_l).
\]

Решим наше разностное уравнение методом разделения переменных, будем искать решение в виде
\[
	u_l^k = \lambda^k e^{imx}, \quad x = x_l = lh.
\]
Подставим:
\[
	\dfrac{\lambda^{k + 1}e^{imx} - \lambda^{k}e^{imx}}{\tau} = \dfrac{\lambda^k e^{im(x+h)} - 2\lambda^k e^{imx} + \lambda^k e^{im(x-h)}}{h^2}.
\]
Несложными выкладками отсюда находится
\[
	\lambda = 1 + 2\sigma \big(\cos mh - 1\big)\,, \quad \sigma = \dfrac{\tau}{h^2}.
\]
Но такое решение не удовлетворяет граничным условиям; можно рассмотреть какую-нибудь комбинацию решений! Заметим, что $\lambda(m)$~--- чётная функция, поэтому 
\[
	\lambda^k(m) \big(e^{imx} - e^{-imx}\big) = 2i\lambda^k(m)\sin(mx)
\]
тоже решение. Оно удовлетворяет граничным условиям при целых $m$; в итоге получаем
\[
	\boxed{u_l^k = \lambda^k(m) \cdot \sin (mx), \quad m \in \Z}\,.
\]
У нас теперь есть $n-1$ ЛНЗ решение, из которых можно собирать новые:
\[
	\varphi(x) = \sum\limits_{m = 1}^{n - 1} C_m \lambda^k(m) \cdot \sin (mx).
\]

\begin{rem}
	Остальные значения $m$ нам не интересны, поскольку у нас набралась $n-1$ базисная функция: действительно, изначально наши разностные уравнения решались однозначно, а сейчас мы их решили, учитывая граничные условия, но отпустив начальные. А их как раз $n - 1$~--- от $u^0_1$ до $u^0_{n-1}$, они и создают все степени свободы.
\end{rem}

Рассмотрим $\tau = h^2 \so \sigma = 1$:
\[
	\lambda(m) = -1 + 2 \cos mh.
\]
При $m = n - 1$ и густой сетке (большом $n$)
\[
	\cos \dfrac{(n - 1)\pi}{n} = \cos \left(\pi - \dfrac{\pi}{n}\right) \approx -1 \so \lambda(n - 1) \approx -3!
\]

При увеличении $k$ решение
\[
	(-3)^k \sin(n - 1)x
\]
очень быстро растёт по модулю и всё время меняет знак. Кажется, что-то пошло не так!

\begin{rem}
	Реальное решение такой задачи~--- быстро убывающая колебашка. Конечно, пространственный шаг взят большим: у начальных данных есть переменность на том же масштабе. Однако то, что при уменьшении шага по времени $k$ получает возможность становиться больше, уже вообще ни в какие ворота не лезет.
\end{rem}

%unsure
%когда говорят про устойчивость, имеют в виду схему с фиксированным \tau или нет?
%unsure

Чтобы решения не вымирали подобным образом, можно наложить ограничение $|\lambda| \leqslant 1$:
\[
	1 + 2\sigma \big(\cos mh - 1\big) \leqslant 1 \so 2\sigma \cos mh - 1 \leqslant 0 \so 2 \sigma \cos mh \leqslant 1.
\]
Чтобы это выполнялось при любых $m$, нужно, чтобы
\[
	\boxed{\sigma \leqslant \dfrac{1}{2} \eqv \tau \leqslant \dfrac{h^2}{2}} \, .
\]

Если точно так же решить разделением переменных систему уравнений для простейшей неявной схемы, получим
\[
	\lambda = \dfrac{1}{1 + 2\sigma(1 - \cos mh)} \leqslant 1,
\]
и устойчивость всегда присутствует.

\paragraph{Общее определение устойчивости, теорема об устойчивости и сходимости}

В начале книги \cite{gavurin} есть общие рассуждения про вычислительные методы и разные пространства. В книге \cite{comp-krilov-2} есть про устойчивость, аппроксимацию, сходимость, их связь между собой, и про схемы для уравнения теплопроводности.

С какой ситуацией мы сталкиваемся, занимаясь сеточными методами? У нас есть оператор $A\col \; U \to F$, и мы решаем уравнение вида
\[
	Au = f.
\]
Выбирая сетку с шагом $h$ на отрезке, мы вместо функций на отрезке начинаем рассматривать функции на самой сетке~ они образуют другое, гораздо более маленькое пространство $U_h$. При этом по любому элементу $U$ можно легко найти элемент $U_h$, просто вычислив его значения на сетке. Аналогично строится пространство $F_h$\footnote{Зачастую $U_h = F_h$ и даже $U = F$, но может быть и не так, в принципе~--- вдруг, например, оператор действует в пространство функций на другом отрезке, или просто там другие ограничения на гладкость/непрерывность.}.

Наконец, есть оператор $A_h \col \; U_h \to F_h$~--- приближение $A$, которое получается при переходе к конечным разностям. Для иллюстрации полезна диаграмма
\[
	\xymatrix{
		U \ar@{->}[r]^{A} \ar@{->}[d]_{\varphi_h} & F \ar@{->}[d]^{\psi_h} \\ 
		U_h \ar@{->}[r]^{A_h} & F_h
	}
\]
\begin{de}
	Операторы $\varphi_h(u)(x_l) = u(x_l)$ и такой же $\psi_h$ называются \ti{операторами (простого) сноса}.
\end{de}

\begin{rem}
	Понятно, что диаграмма должна быть почти коммутативна, но не совсем: если мы сначала продифференцируем функцию, а потом возьмём результат на сетке, и если мы сначала возьмём её на сетке, а потом посчитаем разностный аналог производной, получатся близкие, но разные вещи. Разность
	\[
		A_h\varphi_h(u) - \psi_h(Au)
	\]
	называется \ti{естественной погрешностью метода}.
\end{rem}

Далее, записывается разностное уравнение
\[
	A_h \tilde{u} = \psi_h(f)
\]
и решается.

Во всех четырёх пространствах надо ввести нормы. В пространствах функциональной природы $U$, $F$ они уже и так есть, вероятно.

\begin{de}
	Говорят, что норма на $U_h$ \ti{согласована} с нормой на $U$, если верно, что 
	\[
		\|\varphi_h u\|_{U_h} \to \|u\|_U,
	\]
	когда $h \to 0$ хотя бы для $u \in K \subset U$, где $K$ плотно в $U$.

\end{de}

Будем считать, что у нас нормы согласованы.

\begin{de}
	Говорят, что $A_h$ \ti{аппроксимирует} $A$ на $u \in U$, если 
	\[
		\big\|A_h\varphi_h(u) - \psi_h(Au)\big\| \to 0 \text{ при } h \to 0.
	\]
\end{de}

\begin{de}
	Говорят, что сеточные функции $u_h$ сходятся к функции $u \in U$, если 
	\[
		\big\|u_h - \varphi_h(u)\big\| \to 0 \text{ при } h \to 0.
	\]
\end{de}

\begin{de}
	Говорят, что \ti{сеточное приближение} обладает \ti{свойством аппроксимации}, если $A_h$ аппроксимирует $A$, и сеточные функции $f_h$ сходятся к $f$.
\end{de}

\begin{de}  
	Говорят, что сеточное приближение \ti{устойчиво}, если
	\begin{enumerate}
		\item Уравнение $A_hu_h = f_h$ однозначно разрешимо для всех $f_h \in F_h$;
		\item Для этого решения $\|u_h\| \leqslant k \|f_h\|$, где $k$ не зависит от $h$.
	\end{enumerate}
\end{de}

\begin{thm}[Основная теорема теории разностных методов]
	Пусть дана некоторая краевая задача, и сеточная аппроксимация удовлетворяет следующим свойствам:
	\begin{enumerate}
		\item $u^*$~--- единственное решение уравнения $Lu^* = f$.
		\item Сеточное приближение обладает свойством аппроксимации.
		\item Сеточная задача устойчива.
	\end{enumerate}
	Тогда есть сходимость сеточных решений: $u^*_h \to u$.
	\begin{proof}
		Запишем ошибку сеточного решения:
		\[
			w_h = u_h^* - \varphi_h u^*.
		\]
		Заметим, что по свойству устойчивости
		\begin{align*}
			\|w_h\| &\leqslant k \|L_h w_h\| = k\|L_h u_h^* - L_h \varphi_h u^*\| = \\ &= k\|f_h - \psi_h f + \psi_h f - L_h \varphi_h u^*\| \leqslant k\|f_h - \psi_h f\| + k\|\psi_h L u^* - L_h \varphi_h u^*\|.
		\end{align*}
		Оба слагаемых в правой части стремятся к нулю по свойству аппроксимации.
	\end{proof}
\end{thm}

\paragraph{Разностные схемы для задач с начальными условиями, дискретное преобразование Фурье}

\begin{rem}
	Мне кажется, тут не очень понятно вышло, надо бы потом переписать. 
\end{rem}

В этом параграфе в целом посмотрим на уравнение 
\[
	\dfrac{\pd u}{\pd t} = Lu + f,
\]
где всё многомерное (т.е. $u$~--- вектор, а $L$~--- <<матрица>> из частных производных), и $L$~--- линейный дифференциальный оператор с постоянными коэффициентами. Область определения $u$~--- цилиндр $D\times [0, \, T] \subset \R^{p + 1}$. Начальные условия~--- $u(x, \, 0) = \varphi(x)$.

Ограничимся теперь ситуацией, когда $D$~--- куб $[0, \, 2\pi]^p$, а граничные условия периодические по каждой из переменных (т.е. $u(x_1, \ldots, \, 0, \ldots, \, x_p; \, t) = u(x_1, \ldots, \, 2\pi, \ldots, \, x_p; \, t)$). 

По каждой из пространственных переменных выберем одинаковые шаги
\[
	h = \dfrac{2\pi}{N},
\] 
а по временной~--- шаг
\[
	\tau = \dfrac{T}{M}.
\]
По пространственной причём рассматриваем только от $0$ до $M-1$, потому что справа снова будет то же граничное значение. Уравнения будут двухслойными, с $k$-го и $k+1$-го слоя.

Даже в неявном случае с помощью разностной прогонки можно выразить все следующие слои через предыдущие и получить уравнения
\[
	u_h(k + 1) = R_h u_h(k) + \rho_h(k),
\]
где $R_h$~--- \ti{оператор перехода в однородном случае}. $R_h$~--- просто матрица с постоянными коэффициентами, а $\rho_h(k)$ зависит от $f$.

\begin{rem}
	Всё-таки скажу про эти обозначения пространств... $V_h$~--- пространство сеточных функций \ti{на фиксированном слое} (т.е. оно $N$-мерное), а $F_h$~--- видимо, аналогичное пространство, которое мы отличаем только по формальным причинам, в котором лежат $f_h$~--- сеточные версии $f$. Нормы в обоих пространствах~--- просто $l^{\infty}$, т.е.
	\[
		\big\|\{u_i\}\big\| = \max |u_i|.
	\]
\end{rem}

\begin{thm} \label{thm:stab-1}
	Для устойчивости при $f = 0$ необходимо и достаточно, чтобы были ограничены $\|R_h^k\|$ (здесь $k$~--- степень!) при $k\tau \leqslant T$.
	\begin{proof}
		Оно в целом понятно: когда нет $f$-ок, нет и $\rho$-шек, а без них переход на следующий слой~--- тупо умножение на матрицу $R_h$. Ясно, что если нормы этих матриц в совокупности ограничены, то и
		\[
			\|u_h(k+1)\| \leqslant C \|\varphi\|,
		\]
		где $\varphi$ задаёт начальные условия.

		Обратно тоже понятно: если ограниченности норм матриц нет, можно просто пойти от противного и сконструировать мерзкую последовательность.
	\end{proof}
\end{thm}

\begin{thm}
	Если $f \neq 0$ и $\|R_h^k\|$ ограничены, то для устойчивости достаточно, чтобы
	\[
		\|\rho_h\|_{V_h} \leqslant c_2 \tau \|f_h\|_{F_h}
	\]
	\begin{proof}
		Тут несложная оценка, она есть в бумажном конспекте. 
	\end{proof}
\end{thm}

\begin{cor}
	Если $\|R_h\| \leqslant 1 + c_3\tau$, то есть устойчивость (при $f = 0$).
	\begin{proof}
		\[
			\|R_h^k\| \leqslant \|R_h\|^k \leqslant (1 + c_3 \tau)^k \leqslant e^{c_3 \tau k} \leqslant e^{c_3 \tau}.
		\]
	\end{proof}
\end{cor}

По поводу этих теорем можно ещё заглянуть в следующий параграф \ref{par:neumann}, там доказаны очень похожие вещи.

Перейдём теперь к дискретному преобразованию Фурье. Пусть размерность $p$ пока равна $1$. Введём на пространстве $V_h$ функций на фиксированном слое скалярное произведение:
\[
	(u_h, \, v_h) = h \sum\limits_{i = 0}^{N - 1} u_h \ov{-}{v_h}
\]

\begin{st}
	Набор функций $e_m(x) = e^{imx}$, где $m \in 0\ldots N-1$, образует ортогональный базис в $V_h$, причём
	\[
		(e_m, \, e_m) = 2\pi.
	\]
	\begin{proof}
		Чтобы увидеть, что они ортогональны, достаточно посчитать скалярное произведение. Отсюда следует, в принципе, что они ЛНЗ. Ну а дальше~--- их $N$, пространство $N$-мерное, потому и базис.
	\end{proof}
\end{st}

\begin{de}
	\ti{Обратное дискретное преобразование Фурье}~---
	\[
		\{a_1, \, \ldots, \, a_N\} \mapsto \sum\limits_{i = 1}^{N-1} a_i e_i(x).
	\]
	\ti{Прямое ДПФ}~---
	\[
		u_h \mapsto \dfrac{1}{2\pi}\big\{(u_h, \, e_1), \ldots, \, (u_h, \, e_n)\big\}.
	\]
	Ясно, что это взаимно обратные операторы.
\end{de}

\begin{st}[Формула замкнутости]
	Дискретное преобразование Фурье~--- почти унитарный оператор, т.е. 
	\[
		\left(\sum\limits_{i = 1}^{N - 1} a_i e_i, \; \sum\limits_{i = 1}^{N - 1} b_i e_i\right) = 2 \pi \sum_{i = 0}^{N - 1} a_i \ov{-}{b_i}.
	\]
	\begin{proof}
		Проверяется прямым вычислением.
	\end{proof}
\end{st}

\begin{st}
	$e^{imx}$~--- собственная функция оператора сдвига
	\[
		T_h u(x) = u(x + h)
	\]
	с собственным числом $e^{imh}$.
	\begin{proof}
		Действительно,
		\[
			T_h e^{imx} = e^{im(x + h)} = e^{imh} e^{imx}.
		\]
	\end{proof}
\end{st}

\begin{rem}
	У нас периодические граничные условия, поэтому оператор сдвига может действовать, <<переходя>> через границу:
	\[
		T_h \{u_0, \, \ldots, \, u_{N - 1}\} = \{u_{1}, \, u_2, \, \ldots, \, u_{N-1}, \, u_0\}.
	\]
	Можно представлять себе, что индекс $i$ на самом деле меняется от $-\infty$ до $\infty$, но $u_{i + N} = u_i$. Периодическую функцию можно восстановить, зная её значения внутри периода, вот и здесь так же.
\end{rem}

\begin{rem}
	В такой ситуации любой разумный разностный оператор можно собрать из операторов сдвига. Например, пусть
	\[
		(Du)_i = \dfrac{u_{i + 1} - 2u_i + u_{i - 1}}{h}.
	\]
	Это можно переписать просто как
	\[
		Du = \dfrac{T_hu - 2u + T_{-h}u}{h}.
	\]
	Общая формула, естественно, будет такая:
	\[
		Lu = \sum\limits_{\alpha} c(\alpha) T_h^{\alpha}(u),
	\]
	где $\alpha \in \Z$~--- показатель степени.

	Тем удобнее будет применять эти операторы к экспонентам~--- от них ведь сдвиг считать легко.
\end{rem}

Это всё можно написать и в многомерии, при $p>1$, но, кажется, У Оли этого нет. См. конспект Ангелины, билет и так длинный.

Применим теперь построенную теорию к сеткам. Сеточное уравнение будет выглядеть примерно так:
\[
	\sum\limits_{\alpha \in A_0} A(\alpha) u(x + \alpha h, \, k \tau) = \sum\limits_{\beta \in B_0} B(\beta) u\big(x + \beta h, \, (k+1) \tau\big).
\]
Ну, это просто два слоя, $k$-й и $k+1$-й. Теперь сделаем ДПФ, пусть
\[
	u(x, \, k\tau) = \sum a^k(m) e^{imx}.
\]
Подставим это в суммы:
\[
	\sum\limits_{\alpha \in A_0} e^{im\alpha h} A(\alpha) \sum\limits_m a^k(m) e^{imx} = \sum\limits_{\beta \in B_0} e^{im\beta h} B(\beta) \sum\limits_m a^{k+1}(m) e^{imx}.
\]
Слева и справа написаны два разложения по базису, коэффициенты в которых должны совпадать:
\[
	a^k(m) \sum\limits_{\alpha \in A_0} e^{im\alpha h} A(\alpha)  = a^{k+1}(m) \sum\limits_{\beta \in B_0} e^{im\beta h} B(\beta) 
\]
В итоге получаем
\[
	a^{k + 1}(m) = c(m) a^k(m), \quad c(m) = \dfrac{\sum\limits_{\alpha \in A_0} e^{im\alpha h} A(\alpha)}{\sum\limits_{\beta \in B_0} e^{im\beta h} B(\beta)}.
\]

\paragraph{Необходимое условие устойчивости по фон Нейману}\label{par:neumann}

\begin{rem} 
	%unsure
	Коэффициент $c$~--- по сути, видимо, диагональная матрица. Это матрица перехода между слоями в терминах коэффициентов Фурье. 

	Я не совсем понял, где в конспекте проходит грань между матрицей и числом (всё усложняется тем, что в высших размерностях $m$~--- мультииндекс, а $u^k$ и $c$, видимо, <<тензоры>>). Поэтому я буду исходить из того, что $c(m)$~--- просто число, а $c$~--- набор этих чисел, причём
	\[  
		\|c\| = \|c\|_{l^{\infty}} = \max\limits_m c(m).
	\]
	Ну и вообще, пусть все доказательства будут одномерными.
	%unsure
\end{rem}

\begin{thm}
	Для устойчивости при $f = 0$ необходимо и достаточно, чтобы
	\[
		\|c^k\| \leqslant c_3, \quad k\tau \leqslant T.
	\]
	\begin{proof}
		Интересно, видимо, доказывать достаточность. Попробуем просто найти оценку на норму $u_h(k)$.
		\[
			a(k, \, m) = c^k(m)a(0, \, m),
		\]
		поэтому 
		\[
			u_h(k) = \sum\limits_m a(k, \, m) e^{imx} = \sum\limits_m a(0, \, m) c^k(m) e^{imx}.
		\]
		Далее $K$~--- произвольная неотрицательная константа, в которую можно вносить другие.
		По формуле замкнутости
		\begin{align*}
			\big\|u_h(k)\big\|_{l^2}^2 &= 2\pi \sum\limits_m \big|a(0, \, m) c^k(m)\big|^2 \leqslant \\ & \leqslant K \max\limits_{m} \big|a(0, \, m)\big|^2 \big|c^k(m)\big|^2 \leqslant  K \max\limits_m |a(0, \, m)\big|^2.
		\end{align*}
		Последний переход возможен, поскольку $\|c^k\| \leqslant c_3$.
		При этом
		\begin{align*}
			\max\limits_m |a(0, \, m)\big|^2 &= \left(\max\limits_m |a(0, \, m)\big|\right)^2 = \|a^0\|_{l^{\infty}}^2 \leqslant \\ &\leqslant K\|a^0\|_{l^2}^2 = K \big\|u_h(0)\big\|_{l^2}^2 \leqslant K\big\|u_h(0)\big\|_{l^{\infty}}^2.
		\end{align*}
		В итоге получаем
		\[
			\big\|u_h(k)\big\|_{l^{\infty}}^2 \leqslant K\big\|u_h(k)\big\|_{l^2}^2 \leqslant K\big\|u_h(0)\big\|_{l^{\infty}}^2.
		\]
		Это и есть устойчвость, по сути.
		Чтобы доказать необходимость, предположим, что нет такой оценки $\|c^k\| \leqslant c_3$, не зависящей от $h$ и $\tau$. Рассмотрим $u_h(0) = e^{imx}$. Тогда
		\[
			a(0, \, l) = \delta_{ml} \so u_h(k) = c^k(m)e^{imx}.
		\]
		По предположению мы можем так подобрать $h, \, \tau, \, m, \, k$, что $\big|c^k(m)\big|$ станет сколь угодно большим; но тогда это произойдёт и с $\|u_h(k)\|$! Понятно, что никакой устойчивости нет и в помине.
	\end{proof}
\end{thm}

\begin{thm}[условие фон Неймана]
	Для устойчивости при $f = 0$ необходимо и достаточно, чтобы собственные числа $c$ удовлетворяли условию
	\[
		|\lambda| \leqslant 1 + c_4 \tau.
	\]
	\begin{proof}
		Ну, достаточность не слишком сложна:
		\[
			\|c^k\| = \max \limits_m \big|c(m)\big|^k \leqslant \big|1 + c_4 \tau\big|^k \leq e^{kc_4\tau} \leqslant e^{c_4 T}.
		\]
		Необходимость, впрочем, тоже. Пусть этого условия нет; тогда для любого $c_4$ можно подобрать такие $h$, $\tau$ и $m$, что $c(m) > 1 + c_4\tau$. Но тогда 
		\[
			\|c^k\| \geqslant \big|c(m)^k\big| \leqslant (1 + c_4 \tau)^k.
		\]
		Ясно, что увеличивая $c_4$, можно неограниченно увеличивать $\|c^k\|$.
	\end{proof}
\end{thm}

\begin{rem} 
	Кажется, в многомерии это условие только необходимое.
\end{rem}

\paragraph{Простейшие схемы для уравнения бегущей волны}

\begin{de}
	\ti{Уравнение бегущей волны} имеет вид
	\[
		\dfrac{\pd u}{\pd t} = a \dfrac{\pd u}{\pd x}.
	\]
\end{de}

\begin{rem}
	Любая функция вида
	\[
		u(x, \, t) = f(x + at)
	\]
	является решением.
\end{rem}

\begin{rem}
	Видимо, мы будем работать с периодическим граничными условиями I типа, как и с простейшим уравнением теплопроводности.
\end{rem}

Запишем разностные уравнения:
\[
	\dfrac{u^{k + 1}_l - u^k_l}{\tau} = a \dfrac{u^k_{l+1} - u^k_l}{h}.
\]
Это обычная, явная схема. Чтобы исследовать устойчивость, найдём матрицу перехода. Для этого подставим $u_l^k = e^{imx}$ и $u_l^{k + 1} = c(m)e^{imx}$:
\[
	\dfrac{c(m)e^{imx} - e^{imx}}{\tau} = a \dfrac{e^{imx}e^{imh} - e^{imx}}{h}.
\]
Отсюда быстро находим
\begin{equation}\label{eq:wave-trans}
	\boxed{c(m) = 1 + a\sigma(e^{imh} - 1), \quad \sigma = \dfrac{\tau}{h}}\,.
\end{equation}

Нужно понять, когда $\big|c(m)\big| \leqslant 1$.

%unsure
% \begin{rem}
% 	Этого будет достаточно, но мне не очень понятно, почему не проверяется более точное условие с $1 + c\tau$... Впрочем, в данном конкретном случае это, видимо, то же самое.
% \end{rem}
%unsure

\begin{st}
	$\hphantom{.}$
	\begin{enumerate}
		\item Если $a < 0$, то $|c| > 1$.
		\item Если $a > 0$ и $|a\sigma| \leqslant 1$, то $|c| < 1$.
	\end{enumerate}
	\begin{proof}
		Очень советую нарисовать все картинки, иначе непонятно будет. Если кратко, то
		\begin{enumerate}
			\item $e^{imh}$ пробегает единичную окружность.
			\item $e^{imh} - 1$ пробегает единичную окружность с центром в $-1$ (правой стороной она касается нуля).
			\item Умножение на $a\sigma$ либо просто растягивает (относительно нуля), либо растягивает и переворачивает.
			\item Когда $a < 0$, переворачивает, и получается окружность справа от нуля. После прибавления $1$~--- окружность справа от единицы.
			\item Если $a > 0$, получается окружность слева от единицы, которая через неё проходит, радиуса $a\sigma$. Логично, что этот радиус можно увеличить до $1$~--- тогда центр будет в нуле. А дальше нельзя.
		\end{enumerate}
	\end{proof}
\end{st}

\begin{figure}[h]
	\begin{center}
		\includegraphics[scale=0.8]{../img/pde/wave-dep.pdf}
	\end{center}
	\caption{К замечанию \ref{rem:wave-dep}.}
\end{figure}


\begin{rem}\label{rem:wave-dep}
	Заметим, что $u$ постоянна на прямой $x + at = const$. Будем пока считать $a > 0$. Рассмотрим значение $u_l^k$. Оно должно определятся соответствущим значением на прямой $t = 0$:
	\[
		lh + ak\tau = sh \so s = l + a\sigma k.
	\]
	С другой стороны, в нашей схеме значение $u_l^k$ определяется $u_l^{k-1}$ и $u_{l+1}^{k-1}$. Если продолжить этот процесс до $k = 0$, увидим, что $u_l^k$ зависит лишь от $u^0_l\ldots u^0_{l + k}$. Таким образом, чтобы вообще использовать нужное значение из начальных данных, надо
	\[
		s \leqslant l + k \so l + a \sigma k \leqslant l + k \so a \sigma \leqslant 1.
	\]
	Если $a < 0$, то мы вообще не будем использовать это значение, ибо красная прямая будет наклонена в другую сторону.
\end{rem}

Именно с этим связано то, что для $a < 0$ срабатывает схема
\[
	\dfrac{u^{k + 1}_l - u^k_l}{\tau} = a \dfrac{u^k_{д} - u^k_{l-1}}{h}.
\]
В ней информация распространяется в другую сторону, и оценки получаются те же с точностью до знака $a$.

\paragraph{Схема Куранта-Рисса}

Рассмотрим теперь систему
\[
	\dfrac{\pd u}{\pd t} = A \dfrac{\pd u}{\pd x},
\]
где $u$~--- вектор из $\R^p$. Будем считать, что $A$~--- симметричная матрица с постоянными коэффициентами (поэтому у неё все собственные числа вещественны).

Собственные числа матрицы $A$ могут быть разных знаков, это приводит к появлению решений-волн, которые бегут в разные стороны. Это причина, по которой ни одна из простейших схем, вероятно, не будет работать.

Рассмотрим две матрицы: $A_+$ и $A_-$. Первая из них имеет положительные собственные числа такие же, как у $A$, а вместо отрицательных у неё нули. $A_-$ вместо отрицательных собственных чисел $A$ имеет их модули, а вместо отрицательных~--- нули. 

\begin{rem}
	Можно эти две матрицы построить в собственном базисе $A$, а потом вернуть их оттуда назад.
\end{rem}

В итоге $A = A_+ - A_-$. Схема будет устроена так:
\[
	\dfrac{u_l^{k+1} - u_l^k}{\tau} = A_+ \dfrac{u^k_{l + 1} - u^k_l}{h}  - A_- \dfrac{u_l^k - u^k_{l-1}}{h}.
\]
Естественно ожидать от неё хорошего поведения. Такая схема называется \ti{схемой Куранта-Рисса}. Найдём матрицу перехода; для этого вычислим её на
\[
	u_l^k = fe^{imx}, \quad f \text{~--- произвольный постоянный вектор.}
\]
Такой же подстановкой, как в прошлом пункте, находим
\[
	c(m) = I + \sigma \left((e^{imh} - 1)A_+ - (1 - e^{-imh}) A_-\right), \quad \sigma = \dfrac{\tau}{h}.
\]
Теперь нам надо искать собственные числа <<матрицы>> $C$ (мы тут всё-таки вляпались в многомерие, но не сильно). Запишем условие на собственные числа:
\[
	g + \sigma \left((e^{imh} - 1)A_+ - (1 - e^{-imh})A_- \right)g = \lambda g.
\]
Отсюда
\[
	\left((e^{imh} - 1)A_+ - (1 - e^{-imh})A_- \right)g = \dfrac{\lambda - 1}{\sigma} g
\]
Слева стоит линейная комбинация матриц с известными собственными числами, причём там, где у первой ненулевое собственное число, у второй ноль, и наоборот. Поэтому получаем
\[
	\lambda = 1 + \sigma \lambda_{A_{\pm}} (e^{\pm imh} - 1).
\]
Очень похоже на \ref{eq:wave-trans}, но только теперь $a = \lambda_{A_{\pm}}$ и $m = \pm m$.

Ясно, что знак $m$ ни на что не повлияет, и теперь $a > 0$. Второе условие получится аналогичным.

\begin{rem}
	Мы не доказывали достаточность условия для многомерности, но в этом конкретном случае её можно доказать.
\end{rem}

\paragraph{Явная схема для уравнения колебаний струны}

Рассмотрим уравнение колебаний
\[
	\dfrac{\pd^2 u}{\pd t^2} = \dfrac{\pd^2 u}{\pd x^2}.
\]

Вообще, надо было бы перейти к системе уравнений первого порядка, но мы попробуем найти $c(m)$ не переходя~--- вдруг получится!..

Схема
\[
	\dfrac{u_l^{k + 1} - 2u_l^k + u_l^{k-1}}{\tau^2} = \dfrac{u^k_{l+1} - 2u^k_l + u^k_{l-1}}{h^2}.
\]
Положим
\[
	u_l^k = e^{imx}, \quad u_l^{k+1} = ce^{imx}, \quad u_l^{k-1} = c^{-1} e^{imx}.
\]
Выйдет уравнение на $c$:
\[
	\dfrac{c + c^{-1}}{2} = 1 + \sigma^2\big(\cos(mh) - 1\big) = p.
\]
Решая, получим
\[
	c = p \pm \sqrt{p^2 - 1}.
\]
Неудивительно, что нашлись два значения для каждого $m$: на самом деле это собственные числа матрицы перехода, $\lambda_1$ и $\lambda_2$, потому что система второго порядка!

Заметим, что по теореме Виета (ну или руками) 
\[
	\lambda_1 \lambda_2 = 1,
\]
поэтому, чтобы была устойчивость, нам нужно, чтобы $|\lambda_1| = |\lambda_2| = 1$. Это условие выполняется, когда корни комплексны:
\[
	\lambda_{1, \, 2} = p \pm i\sqrt{1 - p^2} \so |\lambda_{1, \, 2}| = 1.
\]
Для этого необходимо, чтобы $|p| < 1$. Ещё оно выполняется, когда корни равны. Чтобы они были равны, нужно $p = \pm 1$, поэтому общее условие:
\[
	|p| \leqslant 1.
\]

Посмотрев на формулу для $p$, поймём, что это гарантированно происходит при 
\[
	\boxed{|\sigma| \leqslant 1}\,.
\]

\begin{rem}
	Можно ещё сказать, что при $\sigma = 1$ на самом деле проявляется слабая неустойчивость, но она не влияет на сходимость.
\end{rem}

\paragraph{Явная и неявная схемы для двумерного уравнения теплопроводности}

\begin{de}
	\ti{Двумерное уравнение теплопроводности}
	\[
		\dfrac{\pd u}{\pd t} = \dfrac{\pd^2 u}{\pd x^2} + \dfrac{\pd^2 u}{\pd y^2}.
	\]
\end{de}

Теперь нам понадобится два индекса внизу:
\[
	u^k_{np} \approx u(nh, \, ph, \, k\tau).
\]

Рассмотрим явную схему
\begin{equation}\label{eq:therm-2-expl}
	\dfrac{u^{k+1}_{np} - u^k_{np}}{\tau} = \dfrac{u^k_{n+1 \; p} - 2u^k_{np} + u^k_{n-1 \; p}}{h^2} + \dfrac{u^k_{n \; p+1} - 2u^k_{np} + u^k_{n \; p - 1}}{h^2}.
\end{equation}

Попробуем найти матрицу перехода. Для этого рассмотрим 
\[
	u^k_{np} = e^{imx} e^{ily}, \quad x = nh, \quad y = ph.
\]
Делая стандартную подстановку и преобразования, получаем
\[
	c(m, \, l) = 1 + 2\sigma\big(cos(mh) - 1\big) + 2\sigma\big(\cos(lh) - 1\big), \quad \sigma = \dfrac{\tau}{h^2}.
\]
Чтобы $\big|c(m, \, l)\big| \leqslant 1$ всегда, нужно
\[
	\boxed{\sigma \leqslant \dfrac{1}{4}}\,.
\]
Получилось в два раза жёсткое условие, чем для одномерного уравнения!

Посмотрим теперь на простейшую неявную схему, в ней левая часть уравнения \ref{eq:therm-2-expl} просто заменится на
\[
	\dfrac{u^{k}_{np} - u^{k-1}_{np}}{\tau}.
\]
Тем же путём находим
\[ 
	c(m, \, l) = \dfrac{1}{1 - 2\sigma\big(cos(mh) - 1\big) - 2\sigma\big(\cos(lh) - 1\big)}.
\]
Видно, что необходимое условие устойчивости выполнено всегда, как и в одномерном случае.

Поговорим о том, как решать систему уравнений для неявной схемы, которая целиком выглядит вот так:
\[
	\dfrac{u^{k}_{np} - u^{k-1}_{np}}{\tau} = \dfrac{u^k_{n+1 \; p} - 2u^k_{np} + u^k_{n-1 \; p}}{h^2} + \dfrac{u^k_{n \; p+1} - 2u^k_{np} + u^k_{n \; p - 1}}{h^2}.
\]
Если её переписать, выйдет
\[
	(1 + 4\sigma) v_{np} - \sigma v_{n + 1\; p} - \sigma v_{n \; p+1} - \sigma v_{n - 1\; p} - \sigma v_{n \; p-1} = \alpha_{np},
\]
где
\[
	v_{np} = u_{np}^k \text{ и } \alpha_{np} = u^{k-1}_{np}.
\]
Граничные условия, как обычно в последнее время, I типа:
\[
	v_{00} = v_{0N} = v_{N0} = v_{NN} = 0.
\]
Поэтому будем рассматривать $n$ и $p$ от $1$ до $N-1$. 

Упорядочим элементы $v$ следующим образом:
\[
	v_{11}, \; v_{12}, \; \ldots, \; v_{1 \; N-1}, \; v_{21}, \; \ldots
\]
Тогда матрица системы (для примера $N=4$) будет выглядеть так:
\[
	\begin{array}{ccc|ccc|ccc}
		1 + 4\sigma & -\sigma & 0 & -\sigma & 0 & 0 & 0 & 0 & 0 \\
		-\sigma & 1 + 4\sigma & -\sigma & 0 & -\sigma & 0 & 0 & 0 & 0 \\
		0 & -\sigma & 1 + 4\sigma & 0 & 0 & -\sigma & 0 & 0 & 0 \\
		\hline
		-\sigma & 0 & 0 & 1 + 4\sigma & -\sigma & 0 & -\sigma & 0 & 0 \\
		0 & -\sigma & 0 & -\sigma & 1 + 4\sigma & -\sigma & 0 & -\sigma & 0 \\
		0 & 0 & -\sigma & 0 & -\sigma & 1 + 4\sigma & 0 & 0 & -\sigma \\
		\hline
		0 & 0 & 0 & -\sigma & 0 & 0 & 1 + 4\sigma & -\sigma & 0 \\
		0 & 0 & 0 & 0 & -\sigma & 0 &  -\sigma & 1 + 4\sigma & -\sigma \\
		0 & 0 & 0 & 0 & 0 & -\sigma & 0 & -\sigma & 1 + 4\sigma
	\end{array}
\]
В общем случае получается $(2N-1)$-диагональная матрица.

Пусть
\[
	v_n = (v_{n1}, \, \ldots, \, v_{n \; N-1}), \quad \alpha_n = (\alpha_{n1}, \, \ldots, \, \alpha_{n \; N-1}).
\]
Тогда можно записать систему, как
\[
	A_n v_{n-1} + B_n v_n + C_n v_{n+1} = \alpha_{n},
\]
где $A_n, \; B_n$ и $C_n$~--- блоки из соответствующей строки.
Дальше можно действовать так же, как обычной прогонкой. Это называется \ti{матричная прогонка}.

К сожалению, обычная прогонка содержит умножения чисел, которые делаются за $O(1)$, и работает $O(N)$, а матричная прогонка содержит умножения матриц, которые делаются за $O(N^3)$, и работает за $O(N^4)$. Долго! Поэтому нам такой неявный метод не подходит.

\paragraph{Схема продольно-поперечной прогонки}

Нужно сделать систему трёхдиагональной. Естественное желание~--- вынести часть переменных в правой части уравнения на соседний слой, чтобы сократить количество тех, что входит с текущего. Эта идея приводит к схеме

\[
	\dfrac{u^{k+1}_{np} - u^{k}_{np}}{\tau} = \dfrac{u^{k+1}_{n+1 \; p} - 2u^{k+1}_{np} + u^{k+1}_{n-1 \; p}}{h^2} + \dfrac{u^k_{n \; p+1} - 2u^k_{np} + u^k_{n \; p - 1}}{h^2}.
\]
Если переписать, получится
\[
	\sigma u^{k+1}_{n-1 \; p} - (1 + 2\sigma) u_{np}^{k+1} + \sigma u^{k+1}_{n+1 \; p} = \alpha_{np}.
\]
Матрица трёхдиагональная, всё замечательно.

Надо проверить на устойчивость. Обычной техникой получаем
\[
	c(l, \, m) = \dfrac{1 - 2\sigma(1 - \cos lh)}{1 + 2\sigma(1 - \cos mh)}.
\]
Чтобы эта штука всегдя была меньше $1$ по модулю, нужно
\[
	\boxed{\sigma \leqslant \dfrac{1}{2}}\,.
\]
Только условная устойчивость!

Можно рассмотреть аналогичную схему
\[
	\dfrac{u^{k+1}_{np} - u^{k}_{np}}{\tau} = \dfrac{u^{k}_{n+1 \; p} - 2u^{k}_{np} + u^{k}_{n-1 \; p}}{h^2} + \dfrac{u^{k+1}_{n \; p+1} - 2u^{k+1}_{np} + u^{k+1}_{n \; p - 1}}{h^2}.
\]
У неё свойства примерно такие же, только $l$ и $m$ меняются местами.

Идея: чередовать схемы I и II:
\[
	2k \xrightarrow{\text{I}} 2k+1 \quad 2k+1 \xrightarrow{\text{II}} 2k+2.
\]
Пусть у чётных слоёв будет номер $k$, у нечётных~--- $k + \nicefrac{1}{2}$. Ну и просто пишем последовательно формулы для I сначала, потом для II, и получаем переход
\[
	k \to k+ \frac{1}{2} \to k+1.
\]

Ясно, что при этом коэффициенты перехода перемножатся:
\[
	C = C_{\text{I}} C_{\text{II}} = \dfrac{1 - 2\sigma(1 - \cos lh)}{1 + 2\sigma(1 - \cos mh)} \cdot \dfrac{1 - 2\sigma(1 - \cos mh)}{1 + 2\sigma(1 - \cos lh)}.
\]
У этой штуки получается $|c| \leqslant 1$ всегда, наступает абсолютная устойчивость.

Такую схему называют \ti{схемой продольно-поперечной прогонки.}

\begin{rem}
	К сожалению, если то же самое сделать для трёхмерного уравнения теплопроводности (а там будет три схемы и разбиение слоя на три подслоя), абсолютной устойчивости не выйдет. Это можно проверить прямым вычислением.
\end{rem}

\begin{rem}
	Более подробно все выкладки можно прочитать в бумажном конспекте, там всё понятно. Ещё там рассказывается про \ti{схему расщепления}, которая всегда работает. Потом есть рассуждения про то, как избавиться от наших общих ограничений вроде периодичности граничных условий, кубической формы области и постоянных коэффициентов.
\end{rem}



\paragraph{Задача Дирихле для двумерного эллиптического уравнения, составление разностных уравений}
\label{par:pde::elldirprobl}

\begin{defn}\label{defn:pde::elldirprobl::leq}
	Рассмотрим уравнение в частных производных 2 порядка
	\[
		Lu = f \that \sum_{i,j}a_{ij} \frac{∂u}{∂x_i\, ∂x_j} + \sum_{i} a_j \pder{u}{x_i} + a u = f
	\]
	Пусть все функции заданы на области $Ω \subset \R^n$.

	Если квадратичная форма, соответствующая $L$ знакоопределена, то 
	\begin{itemize}
		\item $L$ называют эллиптическим оператором.
		\item $Lu = f$ называют эллиптическим уравнением.
	\end{itemize}
\end{defn}

\begin{exmp}
	Уравнение Пуассона: $L = Δ$
	\[
		Δu = f \that \sum_{i}\pder[2]{u}{x_i} = f \qquad a_{ii} = 1 > 0 
	\]
\end{exmp}

Задачу Коши для такого уравнения не поставить~"--- нету выделенной переменной.
Так что будем решать граничные задачи. Пусть $Γ = ∂Ω$
\begin{enumerate}[I]
	\item $\left.u\right|_{Γ} = φ$~"---  задача Дирихле
	\item $\left.\pder{u}{n}\right|_{Γ} = ψ$~"---  задача Неймана
\end{enumerate}

\begin{prop}\label{prop:pde::elldirprobl::max}
	Для эллиптических уравений работает принцип максимума:
	\[
		\max_{Ω \mathbin{\cup} ∂Ω} u = \max_{∂Ω} u 
	\]
\end{prop}
\begin{rem}
	Для разностных уравений он работает лишь если нету перекрёстных членов.
\end{rem}
Рассмотрим пока $n=2$.
Будем решать эти задачи методом сеток
\[
	\begin{aligned}
		Ω_h &= \left\{ (nh, ph) \in \ov-Ω \right\} \\
		Ω_h^0 &= \left\{ (nh, ph)\land ((n\pm1)\,h, ph) \land (nh, (p\pm1)\,h) \in \ov-Ω \right\}\\
		Γ_h &=  Ω_h \setminus Ω_h^0
	\end{aligned}
\]

Как вычислять производные в $Ω_h^0$:
\begin{align*} 
	&\pder[2]{u}{x} &&\longrightarrow & & \begin{matrix}
			& 0  & 0 & 0 \\
		p & \lfrac{1}{h^2}  & \lfrac{-2}{h^2} & \lfrac{1}{h^2} \\
			& 0  & 0 & 0 \\
			&    & n &   \\
	\end{matrix} &&& 
	&\pder[2]{u}{y} &&\longrightarrow& & \begin{matrix}
			& 0  & \lfrac{1}{h^2} & 0 \\
		p & 0  & \lfrac{-2}{h^2} & 0 \\
			& 0  & \lfrac{1}{h^2} & 0 \\
			&    & n &   \\
	\end{matrix} \\[1em]
			&\frac{∂^2u}{∂x\,∂y} &&\longrightarrow& & \begin{matrix}
			& -\lfrac{h^2}{4}  & 0 & \lfrac{h^2}{4} \\
				p & 0  & 0 & 0 \\
					& \lfrac{h^2}{4} & 0 & -\lfrac{h^2}{4}\\
					&    & n &   \\
	\end{matrix}
\end{align*}
Таким образом, видно что шаблон схемы состоит из 9 точек, а если перекрёстных членов нету, то 
из пяти. Будем считать что их всё-таки нету. Можно же привести к 
сумме квадратов.

Запишем разностное уравнение как в первой главе
\begin{equation} \label{eq:pde::elldirprobl::diffeq}
	A_{np} u_{n+1,p} + B_{np} u_{n-1, p} + C_{np} u_{n, p} + D_{np} u_{n, p+1} + E_{np} u_{n,p-1}  
	= f_{np}
\end{equation}

Нетрудно выразить и его коэффициенты
\[
	\begin{aligned}
		A_{np} &= \frac{a_{11}}{h^2} + \frac{a_1}{2h} & 
		B_{np} &= \frac{a_{11}}{h^2} - \frac{a_1}{2h} \\ 
		C_{np} &= -\frac{2a_{11}}{h^2} - \frac{2a_{22}}{h^2}+ a \\
		D_{np} &= \frac{a_{22}}{h^2} + \frac{a_2}{2h} &
		E_{np} &= \frac{a_{22}}{h^2} - \frac{a_2}{2h} 
	\end{aligned}
\]

Поскольку $a_{11}, a_{22} > 0$ (положительно определённый) $C_{np}$ скорее всего $< 0$
Для сходимости разностных схем мы обычно требовали диагонального преобладания,
а тут это выливается в условие на $a$
\[\def\do#1{{#1}_{np} + }
  \abs{C_{np}} > \abs{\docsvlist{A,B,D} E_{np}} \Leftarrow a < 0
\]

Осталось понять что делать с $Γ_h$.
\begin{enumerate}
	\item Простой снос на границу: выбираем $M\that u_{np} = φ(M)$. Точность тут $O(h)$.
		А нам бы лучше извернуться и сделать $O(h^2)$.
	\item Снос с интерполяцией: найдем поточнее точку пересечения с границей.\par
		Например, между ${n-1}$ и $n$
		\[
      p_1(x) = \frac{x-h(n-1)}{h} \, u_{n,p} + \frac{hn - x}{h}\, u_{n-1,p}
		\]
		Пусть граница пересекается в $n + d$, тогда у нас получается уравнение на $d$
		\[
			\left(1+\frac{d}{h}\right) \, u_{n,p} + \frac{d}{h}\, u_{n-1,p} = φ(M), \quad
      M = h(n+d),hp
		\]
		Вот тут уже $O(h^2)$, первый порядок мы убираем выбором $d$.
\end{enumerate}

% \begin{aux}
%   Можно похожим на снос с интерполяцией метод посчитать вторую производную на граю.
%   По сути меняем шаг в сетке. Но мы этого
%   делать не будем. Надеюсь.
%
%   А вот как решать задачу Неймана мы не разбирали. Вопрос называется задача Дирихле, так
%   пусть кроме неё ничего и не будет
% \end{aux}

\paragraph{Итерационный метод решения сеточной системы}
\label{par:pde::iterell}

Посмотрим на систему \eqref{eq:pde::elldirprobl::diffeq}.
Здесь ровно та же проблема, что и в неявной схеме в уравнении
теплопроводности: прогоночные коэффициенты стали матрицами.  По слоям решать
не выйдет: граница какой угодно формы.\note{на задачу коши наверное можно
посмотреть как на граничное условие на дне цилиндра.} 
А делать $O(N^4)$ как-то не хочется.

\begin{aux}
  Говоря об этом уравнении как о линейной системе, мы имеем в 
  виду что собрали одномерный вектор из $u_{ik}$ просто расположив
  строки друг за другом. Как двумерные массивы в фортране.
\end{aux}

Будем использовать итеративные методы
\[
  Au = f \iff u = Bu + g \qquad u_{n+1} = Bu_n + g 
\]
\begin{prop}[Теорема о сжимающем отображении]
  Если $\norm{B} < 1$ итерации выше сходятся.
\end{prop}

\begin{rem}
  Это достаточное условие. Необходимым и достаточным будет 
  $\max \abs{λ_B} < 1$. Вообще, $\norm{B} = \max \abs{λ_B}$ для симметричных матриц.
\end{rem}
\begin{prop}
    $\displaystyle
      \norm \infty{B} = \max_i \sum_j \abs{b_{ij}} 
    $
\end{prop}

Преобразуем нашу систему к пригодному для итераций виду
\[
  u_{np} = -\frac{1}{C_{np}} \left(  
	p_{np} u_{n+1,p} + B_{np} u_{n-1, p} + D_{np} u_{n, p+1} + E_{np} u_{n,p-1}  \right)
  - \frac{f_{np}}{C_{np}}
\]
Граничное условие тоже можно итеративно решать\note{тут кто-то перепутал знак}
\[
  u_{n,p} = -\frac{d}{h+d}\, u_{n-1,p} + \frac{h}{h+d}\,φ(M)
\]


Докажем, что метод сходится. 
\begin{tproof}
  C граничным условием всё понятно: $\frac{d}{h+d} < \frac{1}{2}$
  а вот с серединкой чуть хитрее
  \begin{enumerate}
    \item $a < 0$. Из диагонального преобладания сходу следует что 
      $\norm\infty{B} < 1$.
    \item $a = 0$. Тут  уже нужно возиться с более точным критерием.

      Пойдём от противного: пусть $\exists\, λ_B \that \abs{λ_B} =1$.
      Пусть $M = \max \abs{v_{np}}$, $Bv = λ_Bv$.
      Обозначим его $\abs{v_{n_0, p_{0}}}$
      Тогда 
      \[
        M = \abs{λ_B} \abs{v_{n_0, p_{0}}}\leqslant 
        \frac{\abs{A_{np}}}{\abs{C_{np}}} \abs{v_{n_0+1, p_0}} +
        \frac{\abs{B_{np}}}{\abs{C_{np}}} \abs{v_{n_0-1, p_0}} +
        \frac{\abs{D_{np}}}{\abs{C_{np}}} \abs{v_{n_0, p_0+1}} +
        \frac{\abs{E_{np}}}{\abs{C_{np}}} \abs{v_{n_0, p_0-1}} \leqslant M
      \]
      Тогда со всей неизбежностью мы получаем что и весь шаблон (крестик)
      равен $M$ по модулю. Таким способом мы неизбежно дойдём до границы
      \[
        \frac{d}{d+h} M = M, \quad \frac{d}{d+h}< \tfrac{1}{2}
      \]
      А того, что выше, не бывает.
  \end{enumerate}
\end{tproof}

\paragraph{Анализ сходимости простейшего итерационного метода для модельной задачи.}

\begin{aux}
  Предупреждение: эти два параграфа пишутся в последние пару часов.
  Качество их весьма сомнительно. Будьте осторожны.
\end{aux}


Рассмотрим уравнение Пуассона в $\R^2$
\[
  Δu = f \iff \pder[2]{u}{x} + \pder[2]{u}{x} = f(x,y)
\]
Поставим граничную задачу Дирихле:
\begin{itemize}
  \item $\ov-Ω = [0;1] × [0;1]$
  \item $\left. u\right|_{Γ} = 0$
\end{itemize}
Зададим квадратную сетку 
\[
	\begin{aligned}
    h &= \frac{1}{N}, & Ω_h &= \left\{ (ih, kh) \in \ov-Ω \right\} \\
	\end{aligned}
\]
Запишем, наконец, разностное уравнение
\[
  \begin{aligned}
    u_{0p} = u_{Np} = 0 &\qquad u_{n0} = u_{nN} = 0 \\
    \frac{u_{n+1,p} - 2u_{np} + u_{n-1, p}}{h^2} + \frac{u_{n, p+1} - 2u_{np} + u_{n,p - 1}}{h^2} &= f_{np}
  \end{aligned}
\]
Решать точно мы его, разумеется, не будем.
Соорудим шаг итераций
\[
  \begin{aligned}
    u_{mp} = \frac{1}4 \left(u_{n+1,p} + u_{n-1, p} + u_{n, p+1} + u_{n,p - 1}\right) 
    - \frac{h^2}{4}\,f_{ik}
  \end{aligned}
\]
В чуть более человеческой форме это выглядит так:
\[
  A_h u_h = f_h \to u_h = Bu_h + g, \quad B = \frac{h^2}{4} A_h + I  
\]
Поскольку $B$ симметричная, процесс сходится в геометрической прогрессии с показателем
\hbox{$\max \abs{λ_B} =: q$}. Так что озаботимся поисками $λ_B$.

Решим уравение на собственные значения $A_h u_h = λ u_h$ методом Фурье (разделения переменных), 
взяв решение сразу в такой форме
\[
  u_{np} = e^{iπm\frac{n}N}\,  e^{iπ\ell\frac{p}N}.
\]
Отсюда несложно получить, что\note{для краткости $λ_A = λ_{A_h}$}
\[
  λ_{ml} = \frac{2}{h^2} \left( \cos \frac{\pi m} N  - 1\right)
  + \frac{2}{h^2} \left( \cos \frac{\pi l}  nN  - 1\right) = λ_A
\]
Отсюда 
\[
  λ_B = \frac{1}{2} \left( \cos \frac {\pi m} N + \cos \frac {\pi l} N\right) 
\]
Оценим:
\[
  \begin{aligned}
    \max {λ_B} &= \cos \frac \pi N < 1 & \max {λ_B} &= \cos \pi \frac{N-1}N > -1 
  \end{aligned}
\]
Короче говоря, $\abs{q} < 1$  и метод славно сходится. Надо только понять как быстро.
\[
  q^n \leqslant ε  \so n \geqslant \frac{\log \frac 1ε}{\log \frac 1q}
\]
Можно говорить, что $\log \frac 1q$~--- это что-то вроде скорости сходимости. 
Оценим её через $h$
\[
  \cos \frac\pi N \approx 1 - \frac {π^2}2\,h^2 \so \log \frac 1q
  \sim \log \left(1+ \frac {π^2}2\,h^2\right) = O(h^2)
\]
Так что для достижения нужной точности потребуется $O(N^2)$ шагов.
Кажется, многовато. И так на шаге $O(N^2)$ операций.

\paragraph{Метод оптимальной верхней релаксации, описание}

Давайте попробуем какое-нибудь улучшить наш итеративный процесс в
сторону улучшения сходимости. Рассмотрим вот такую итеративную схему:
\[
  \begin{aligned}
    \ov~u_{np}^{n+1} &= -\frac{1}{C_{np}} \left(  
    A_{np} u_{n+1,p} + B_{np} u^{n+1}_{n-1, p} + D_{np} u_{n, p+1} + E_{np} u^{n+1}_{n,p-1}  \right)
    - \frac{f_{np}}{C_{np}} \\
      u_{np} &= u_{np} + ω(\ov~u_{ik - u_{ik}}), ω > 0\qquad
      \hbox{(интерполяция\note{а скорее экстраполяция})}
  \end{aligned}
\]
В зависимости от $ω$ схемы называются по-разному:
\begin{description}
  \item[$ω = 1$] Метод Зейделя
  \item[$ω < 1$] Нижняя релаксация
  \item[$ω > 1$] Верхняя релаксация
\end{description}

Какой смысл у такой схемы? Будем обсчитывать всё в сторону увеличения
индексов. На каждом шаге часть узлов мы посчитали вот только что, для
другой части у нас есть приближения с прошлого раза. 
Как-то вот так это выглядит~"--- 
\fbox{$
\begin{smallmatrix}
  & \bullet & \\
 \times &\ast  & \bullet\\
  & \times & \\
\end{smallmatrix}
$}.
Можно вообще хранить всего один массив.

Можно показать, что выбор $ω$ позволяет улучшить скорость сходимости метода.
Собственно, 
\[
  1 < ω_{\text{opt}} < 2,\quad ω_{\text{opt}} = \frac{2}{1 + \sqrt{1-λ_1^2}} \approx
  2 - c_1 h
\]
Здесь $λ_1$ наибольшее по модулю собственное число. Вообще, по идее, можно
подумать что мы решаем вариационную задачу и оптимизировать $ω$. Надеюсь не 
надо нам этого делать. Вообще, как я понял, $ω_{\text{opt}}$ подбирают эмпирически,
сначала на грубой сетке, потом уменьшают шаг и ещё оптимизируют.

Можно ещё увеличить точность, чередуя обходы сетки.
Там вроде $O\left(\sqrt{h}\right)$ получается.

\end{document}


\clearpage

\appendix
\chapter{Введение в функциональный анализ}
\label{chap:funcan}
\documentclass{trlnotes}
\setlayout{hardcopy}
\usepackage{silence}
\WarningFilter{latex}{Reference}
\graphicspath{{../../img/}}

\begin{document}
    \paragraph{Пространства, отображения}
    Бесконечномерные пространства во многом похожи на конечномерные, но есть и различия. Приведём наглядный пример:

    \begin{thm}(Рисса)
        В бесконечномерном пространстве с нормой единичный замкнутый шар не компактен. \footnote{Верно и обратное утверждение: если в нормированном пространстве единичный замкнутый шар компактен, то оно конечномерно.}
        \begin{proof}
            Чтобы доказать, что что-то не компактно, нужно найти там последовательность, у которой нет сходящейся подпоследовательности. Здесь это нетрудно: подойдёт любой счётный ортнормированный набор векторов!

            Представьте себе: у вас есть $n$ единичных ортогональных друг другу векторов. Вы можете добавить ещё один, и ещё, и ещё... Конечно, в такой последовательности не выбрать сходящейся.
        \end{proof}
    \end{thm}

    В том, что касается линейных отображений, тоже есть тонкости. Мы знаем, что любое линейное отображение конечномерных пространств непрерывно и \ti{ограничено} (т.е. образ единичного замкнутого шара при нём ограничен). В бесконечномерном случае это не так! Однако выполняется такое утверждение:

    \begin{st}
        Для нормированных пространств непрерывность и ограниченность линейных отображений равносильны.
    \end{st}

    В реальности многие (особенно определённые на всём пространстве) интересные отображения ограничены.

    \begin{rem}
        Будем все гильбертовы пространства считать \ti{сепарабельными}. Это по сути равносильно тому, что в них есть счётный базис.
    \end{rem}


    \paragraph{Пара фактов про гильбертовы пространства}

    \begin{rem}
        В бесконечномерных пространствах не все подпространства замкнуты; в частности, там бывают всюду плотные подпространства (как, например, многочлены в пространстве непрерывных функций). Об этом не стоит забывать.
    \end{rem}

    Оказывается, в гильбертовых пространствах ортогональные дополнения устроены почти так же, как и в конечномерной ситуации.

    \begin{st}\label{st:hilb-orth-compl}
        Ортогональное дополнение любого множества является замкнутым линейным подпространством. Если $A \subset H$~--- замкнутое линейное подпространство, то $H = A \oplus A^{\perp}$.
    \end{st}

    Этот факт используется для того, чтобы доказать теорему Рисса: линейные функционалы в гильбертовом пространстве~--- просто скалярные умножения на какие-то вектора.

    \begin{thm}[Рисс]\label{thm:rietz-repr}
        Пусть $H$~--- гильбертово пространство. Тогда каждый вектор $e$ задаёт ограниченный функционал $f_e \col \; H \to \C$ по правилу $x \mapsto (x, \, e)$, и каждый ограниченный функционал на $H$ есть $f_e$ для некоторого однозначно определённого вектора $e \in H$. Определённая этим биекция $H \to H^{*}$ есть сопряжённо-линейный изометрический изоморфизм нормированных пространств.
    \end{thm}

    \paragraph{Спектр оператора}

    Ещё одно различие, не столь наглядное, но очень важное, связано со \ti{спектром} оператора.

    \begin{de}
        Пусть $H$~--- гильбертово пространство, $A\col \; H \to H$~--- ограниченный оператор. \ti{Спектром} $A$ называют множество таких $\lambda \in \C$, что оператор $A - \lambda I$ необратим.
    \end{de}
    Понятие спектра тесно связано с собственными числами:
    \begin{de}
        Говорят, что $\lambda \in \C$~--- \ti{собственное число} оператора $A$, если есть такой вектор $v \in H$, что $Av = \lambda v$.
    \end{de}
    Собственные числа можно охарактеризовать в терминах оператора $A - \lambda I$:
    \begin{st}
        $\lambda$~--- собственное число $A$ тогда и только тогда, когда оператор $A - \lambda I$ не инъективен (то есть склеивает какие-то векторы в один).
        \begin{proof}
            Пусть $\lambda$~--- собственное число, $v$~--- собственный вектор. Тогда $(A - \lambda I)v = 0 = A0$, поэтому оператор не инъективен.

            Докажем в обратную сторону. Пусть оператор $A - \lambda I$ не инъективен. Тогда есть вектор из ядра~--- такой, что $(A - 
            \lambda I)v = 0$, т.е. $Av = \lambda v$.
        \end{proof}
    \end{st}

    Отсюда сразу следует утверждение:
    \begin{st}
        Для конечномерных пространств спектр и множество собственных чисел~--- одно и то же.
        \begin{proof}
            Как мы знаем,
            \[
                \text{необратимость} \eqv \text{неинъективность или несюръективность}.
            \]
            Но в конечномерном случае 
            \[
                \text{несюръективность} \so \text{неинъективность}.
            \]
            Это связано с тем, что несюръективный оператор понижает размерность пространства, что вынуждает его склеивать вектора.

            Поэтому необратимость либо сразу влечёт неинъективность, либо сначала влечёт несюръективность, а потом уже неинъективность. Отсюда
            \[
                \text{необратимость} \eqv \text{неинъективность},
            \]
            что и требовалось доказать.
        \end{proof}
    \end{st}

    В бесконечномерном случае всё не так. Из необратимости неинъективность больше не следует, и у оператора появляются два разных способа быть необратимым:

    \begin{enumerate}
        \item Оператор склеивает векторы.
        \item Образ оператора меньше, чем всё пространство.
    \end{enumerate}

    Поэтому спектр оператора $A$ в бесконечномерном пространстве разбивается на собственные числа и те точки, в которых $A - \lambda I$ не является сюръективным (хоть и векторы не склеивает). 

    \begin{rem}
        Это не мифическая ситуация: обычный оператор умножения на координату (т.е. $Af(x) = x f(x)$) в $L^2\big([a, \, b]\big)$ не имеет собственных чисел, но его спектр равен всему отрезку! 

        Когда мы занимались квантовой механикой, мы находили <<собственные вектора>>~--- дельта-функции. То, что они на самом деле не функции и в $L^2$ не лежат~--- свидетельство описанного феномена!
    \end{rem}

    \paragraph{Компактные операторы}

    Обсудим один класс операторов, очень полезный на практике.

    \begin{de}
        Пусть $H$~--- гильбертово пространство, $B$~--- единичный замкнутый шар в нём. Оператор $A\col \; H \to H$ называют \ti{компактным}, если замыкание множества $A(B)$ компактно.
    \end{de}

    \begin{rem}
        На самом деле, компактный оператор переводит любое ограниченное множество в множество с компактным замыканием.
    \end{rem}

    Мы знаем, что даже единичный шар в $H$ не компактен. Это значит, что $A$~--- оператор с очень маленьким образом, он сжимает всё пространство во что-то крохотное! Это объясняет простоту (и близость к конечномерию) свойств компактных операторов.

    \begin{st}
        Если операторы $A_n$ компактны и $\|A_n - A\| \to 0$, то оператор $A$ компактен.
    \end{st}

    \begin{cor}
        Если операторы $A_n$ конечного ранга (т.е. их образы конечномерны), и $\|A - A_n\| \to 0$, то оператор $A$ компактен.
    \end{cor}

    Главный пример компактного оператора~--- \ti{интегральный оператор}.

    \begin{exm}
        Пусть $\square = [a, \, b] \times [a, \, b]$. Рассмотрим оператор $A$ на $L^2\big([a, \, b]\big)$, действующий по правилу
        \[
            Af(x) = \int\limits_a^b K(x, y) f(y) \, \mathrm{d}y,
        \]
        где $K \in L^2(\square)$. Такой оператор называют \ti{интегральным}, а функцию $K$ называют его \ti{ядром}. В принципе, вместо $L^2$ можно жить в $C$~--- пространстве непрерывных функций, но оно не гильбертово.   
    \end{exm}

    \begin{st}
        Интегральный оператор компактен.
        \begin{proof}[Почти доказательство]
            Разложим функцию $K$ по базису (так можно, правда):
            \[
                K(x, \, y) = \sum\limits_{n, \, m = 0}^{\infty} c_{nm} e_n(x) e_m(y).
            \]
            Рассмотрим последовательность интегральных операторов $A_N$ с ядрами
            \[
                K_N(x, \, y) = \sum\limits_{n, \, m = 0}^{N} c_{nm} e_n(x) e_m(y).
            \]
            Простым преобразованием находим, что
            \[
                A_N f(x) = \sum\limits_{n = 1}^N \left(\,\sum\limits_{m = 1}^N c_{nm} \int\limits_a^b e_m(y) f(y) \, \del y\right) e_n(x).
            \]

            Образ оператора $A_N$ находится внутри линейной оболочки векторов $e_1, \, \ldots, \, e_N$! Это значит, что наш оператор $A$ приближается операторами конечного ранга, а потому компактен.
        \end{proof}
    \end{st}

    \paragraph{Спектры компактных операторов}

    Спектр компактного оператора обладает замечательным свойством:

    \begin{st}
        Пусть $A$~--- компактный оператор. Для любого $\delta > 0$ множество собственных чисел $A$ таких, что $|\lambda| \geqslant \delta$ конечно. Собственное пространство любого $\lambda \neq 0$ конечномерно. 
    \end{st}

    Спектр произвольного самосопряжённого оператора, с другой стороны, обладает такими свойствами:

    \begin{st}
        $\hphantom{.}$
        \begin{enumerate}
            \item Собственные значения самосопряжённого оператора вещественны.
            \item Собственные векторы самосопряжённого оператора, отвечающие разным собственным значениям, ортогональны.
        \end{enumerate}
    \end{st}

    Для операторов, одновременно компактных и самосопряжённых, удаётся доказать вариант \ti{спектральной теоремы}~--- бесконечномерного аналога утверждения о том, что симметричную матрицу можно привести к диагональному виду:

    \begin{thm}[Гильберта-Шмидта] \label{thm:hilb-sch}
        Пусть $A$~--- компактный и самосопряжённый оператор в гильбертовом пространстве $H$. Существует ортогональный базис $\{e_i\}$, состоящий из собственных векторов $A$.
    \end{thm}

    \paragraph{Альтернатива Фредгольма}

    \begin{de}
        \ti{Фредгольмовым} называют такой оператор $T$ на гильбертовом пространстве, что $T = I - A$, где $A$ компактен. 
    \end{de}

    \begin{st}
        Сопряжённый к компактному оператор компактен.
    \end{st}

    \begin{thm}[Альтернатива Фредгольма]\label{thm:fred-alt}
        $\hphantom{.}$
        \begin{enumerate}
            \item Уравнение $T\varphi = f$ разрешимо тогда и только тогда, когда $f$ ортогонально любому решению уравнения $T^{*} \psi_0 = 0$.
            \item Либо уравнение $T\varphi = f$ имеет при любом $f$ ровно одно решение, либо уравнение $T \varphi_0 = 0$ имеет ненулевое решение.
            \item Уравнения $T^*\psi_0 = 0$ и $T\varphi_0 = 0$ имеют одно и то же конечное число линейно независимых решений.
        \end{enumerate}
    \end{thm}

    \begin{rem}
        Эту теорему называют \ti{альтернативой}, потому что трудно вынести безальтернативность приближения сдачи вычей. Представьте себе, что вы смотрите на уравнение $T\varphi = f$. Есть два варианта:
        \begin{enumerate}
            \item Уравнение $T \varphi_0$ не имеет ненулевых решений, и ваша задача разрешима единственным способом. Всё прекрасно!
            \item Оно их таки имеет, и всё не столь прекрасно.
        \end{enumerate}
        Пусть вы попали во второй вариант. Снова выбор:
        \begin{enumerate}
            \item $f$ ортогонально всем решениям уравнения $T^{*} \psi_0 = 0$ (которые теперь уже точно есть по третьему пункту). Тогда ваша задача разрешима, но не одним способом (видимо, их будет бесконечно много).
            \item $f$ не такое. Тогда ваша задача неразрешима.
        \end{enumerate}
    \end{rem}
\end{document}
% vim:wrapmargin=3


% \chapter{Обозначения}
% \input{tex/nomenclature}

\nocite{*}
\printbibliography[
  heading=bibintoc,
  title={Использованная литература}
]

\end{document}

